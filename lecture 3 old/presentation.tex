\documentclass[a4paper, 11pt]{beamer}

\usepackage{polski}
\usepackage[utf8]{inputenc}
\usepackage{units}

\mode<presentation> {
	\usetheme{Frankfurt}
	\setbeamercovered{transparent}
	\usecolortheme{default}
}

\title{Ekonometria Finansowa}
\subtitle{Wielowymiarowe modele szeregów czasowych}
\author{mgr Paweł Jamer\thanks{pawel.jamer@gmail.com}}
\institute[KEiS SGGW]{
	Doktorant, Katedra Ekonometrii i Statystyki SGGW\newline
	Ekspert ds. Modelowania Danych, Polskie Technologie\newline
	Konsultant Zewnętrzny, Polkomtel
}

\begin{document}

	\begin{frame}
		\titlepage
	\end{frame}
	
	\section{Model VAR}
	
	\begin{frame}{Model VAR}
		\begin{block}{\textbf{Model Wektorowej Autoregresji (VAR)}}
			Niech dany będzie szereg czasowy $\boldsymbol{Y}_{t} = 
			\left[Y_{1,t}, Y_{2,t}, \ldots, Y_{n,t}\right]^{\prime}.$ 
			Modelem wektorowej autoregresji o $p$ opóźnieniach szeregu 
			$\boldsymbol{Y}_{t}$ nazwiemy model opisany równaniem
			\[
				\boldsymbol{Y}_{t} =
					\sum_{i=1}^{p}
						\boldsymbol{A}_{i} \boldsymbol{Y}_{t-i} +
						\boldsymbol{\epsilon}_{t},
			\]
			gdzie
			\begin{itemize}
				\item $\mathbb{E}\left(\boldsymbol{\epsilon}_{t}\right) = 
					\boldsymbol{0},$
				\item $\mathbb{E}\left(
					\boldsymbol{\epsilon}_{t}
					\boldsymbol{\epsilon}_{t}^{\prime}
				\right) = \boldsymbol{\Omega}$ --- macierz dodatnio 
				określona,
				\item $\mathbb{E}\left(
					\boldsymbol{\epsilon}_{t}
					\boldsymbol{\epsilon}_{s}^{\prime}
				\right) = \boldsymbol{0}$ dla $t \neq s.$
			\end{itemize}
		\end{block}
		\begin{alert}{\textbf{Uwaga.}}
			Model wektorowej autoregresji o $p$ opóźnieniach przyjęło 
			się oznaczać symbolem $\mbox{VAR}\left(p\right).$
		\end{alert}
	\end{frame}
	
	\begin{frame}{Właściwości}
		Model $\mbox{VAR}:$
		\begin{itemize}
			\item ma często dobre właściwościami prognostyczne i 
				symulacyjne,
			\item dopuszcza pełną dowolność co do wartości parametrów,
			\item uwzględnia występowanie zależności pomiędzy zmiennymi,
			\item traktuje wszystkie zmienne jako objaśniane oraz 
				objaśniające,
			\item nie jest związany z żadną konkretną teorią 
				ekonomiczną,
			\item wymaga oszacowania wielu parametrów.
		\end{itemize}
	\end{frame}
	
	\begin{frame}{Reprezentacja VAR(1)}
		Rozpatrzmy proces VAR(p) \[
			\boldsymbol{Y}_{t} =
				\sum_{i=1}^{p}
					\boldsymbol{A}_{i} \boldsymbol{Y}_{t-i} +
					\boldsymbol{\epsilon}_{t}.
		\]
		Proces ten zapisać możemy jako \[
			\left[\begin{matrix}
				\boldsymbol{Y}_{t}\\
				\boldsymbol{Y}_{t-1}\\
				\vdots\\
				\boldsymbol{Y}_{t-\left(p-1\right)}
			\end{matrix}\right] = \left[\begin{matrix}
				\boldsymbol{A}_{1} & 
				\cdots & \boldsymbol{A}_{p-1} & \boldsymbol{A}_{p}\\
				\boldsymbol{I} & 
				\cdots & \boldsymbol{0} & \boldsymbol{0}\\
				\vdots & \ddots & \vdots & \vdots\\
				\boldsymbol{0} & 
				\cdots & \boldsymbol{I} & \boldsymbol{0}
			\end{matrix}\right] \left[\begin{matrix}
				\boldsymbol{Y}_{t-1}\\
				\boldsymbol{Y}_{t-2}\\
				\vdots\\
				\boldsymbol{Y}_{t-p}
			\end{matrix}\right] + \left[\begin{matrix}
				\boldsymbol{\epsilon}_{t}\\
				\boldsymbol{0}\\
				\vdots\\
				\boldsymbol{0}
			\end{matrix}\right],
		\]
		a zatem jako proces VAR(1).
	\end{frame}
	
	\begin{frame}{Model VMA}
		\begin{block}{\textbf{Model Wektorowej Średniej Ruchomej (VMA)}}
			Niech dany będzie szereg czasowy $\boldsymbol{Y}_{t} = 
			\left[Y_{1,t}, Y_{2,t}, \ldots, Y_{n,t}\right]^{\prime}.$ 
			Modelem wektorowej średniej ruchomej o $q$ opóźnieniach szeregu 
			$\boldsymbol{Y}_{t}$ nazwiemy model opisany równaniem
			\[
				\boldsymbol{Y}_{t} =
					\boldsymbol{\epsilon}_{t} + 
					\sum_{i=1}^{q}
						\boldsymbol{M}_{i} \boldsymbol{\epsilon}_{t-i},
			\]
			gdzie
			\begin{itemize}
				\item $\mathbb{E}\left(\boldsymbol{\epsilon}_{t}\right) = 
					\boldsymbol{0},$
				\item $\mathbb{E}\left(
					\boldsymbol{\epsilon}_{t}
					\boldsymbol{\epsilon}_{t}^{\prime}
				\right) = \boldsymbol{\Omega}$ --- macierz dodatnio 
				określona,
				\item $\mathbb{E}\left(
					\boldsymbol{\epsilon}_{t}
					\boldsymbol{\epsilon}_{s}^{\prime}
				\right) = \boldsymbol{0}$ dla $t \neq s.$
			\end{itemize}
		\end{block}
		\begin{alert}{\textbf{Uwaga.}}
			Model wektorowej średniej ruchomej o $q$ opóźnieniach
			przyjęło się oznaczać symbolem $\mbox{VMA}\left(q\right).$
		\end{alert}
	\end{frame}
	
	\begin{frame}{Reprezentacja VMA($\infty$)}
		Dowolny model wektorowej autoregresji przedstawić można w postaci
		modelu wektorowej średniej ruchomej o nieskończonej liczbie
		opóźnień \[
			\boldsymbol{Y}_{t} =
				\boldsymbol{\epsilon}_{t} + 
				\sum_{i=1}^{\infty}
					\boldsymbol{M}_{i} \boldsymbol{\epsilon}_{t-i}.
		\]
		W szczególności model $\mbox{VAR}\left(1\right)$ przyjmuje postać \[
			\boldsymbol{Y}_{t} =
				\boldsymbol{\epsilon}_{t} + 
				\sum_{i=1}^{\infty}
					\boldsymbol{A}^{i} \boldsymbol{\epsilon}_{t-i},
		\]
	\end{frame}
	
	\begin{frame}{Stabilność}
		\framesubtitle{Definicja}
		\begin{block}{\textbf{Stabilność}}
			Model VAR nazywamy stabilnym, jeżeli \[
				\lim_{k\rightarrow\infty}\left(\boldsymbol{A}^k\right) = 
					\boldsymbol{0},
			\] gdzie $\boldsymbol{A}$ jest macierzą parametrów reprezentacji
			$\mbox{VAR}\left(1\right)$ analizowanego modelu VAR.
		\end{block}
		\begin{alert}{\textbf{Intuicja.}}
			Wpływ zaburzenia $\boldsymbol{\epsilon}_t$ na
			$\boldsymbol{Y}_t$ wygasa w miarę czasu.
		\end{alert}
	\end{frame}
	
	\begin{frame}{Stabilność}
		\framesubtitle{Twierdzenie}
		\begin{block}{\textbf{Twierdzenie}}
			Model $\mbox{VAR}\left(p\right)$ jest stabilny, gdy \[
				\det\left(\boldsymbol{I} - \sum_{i=1}^{p}\boldsymbol{A}_i z^{i}\right) \neq 0
			\] dla $\left| z \right| \leq 0.$
		\end{block}
	\end{frame}
	
	\begin{frame}{Stabilność}
		\framesubtitle{Testowanie}
		\textbf{Rekursywna estymacja parametrów.}\\
			Szacuje się parametry modelu VAR na podstawie pierwszych $\tau$ obserwacji,
			gdzie $\tau$ przyjmuje kolejno wartości $T_0, T_0 + 1, \ldots, T.$ Jako $T_0$
			wybiera się najmniejszą wartość dla której możliwe jest zbudowanie modelu.
			Następnie rysuje się wykresy szeregów uzyskanych oszacowań wraz ze średnimi
			błędami szacunku.
		\\~\\
		\textbf{Testy statystyczne.}
		\begin{itemize}
			\item Test CUSUM.
			\item Test Chowa.
		\end{itemize}
	\end{frame}
	
	\begin{frame}{Rozszerzenia modelu VAR}
		\framesubtitle{Zmienne deterministyczne}
		W równaniu modelu VAR często uwzględnia się zmienne deterministyczne
		(np. stałą, trend, sezonowość): \[
			\boldsymbol{Y}_{t} =
				\boldsymbol{A}_0 \boldsymbol{D}_t +
				\sum_{i=1}^{p}
					\boldsymbol{A}_{i} \boldsymbol{Y}_{t-i} +
					\boldsymbol{\epsilon}_{t},
		\] gdzie
		\begin{itemize}
			\item $\boldsymbol{D}_t$ --- macierz zmiennych
			deterministycznych,
			\item $\boldsymbol{A}_0$ --- macierz parametrów
				związanych z $\boldsymbol{D}_t.$
		\end{itemize}
	\end{frame}
	
	\begin{frame}{Rozszerzenia modelu VAR}
		\framesubtitle{Zmienne egzogeniczne}
		Innym typem często uwzględnianych w modelu VAR zmiennych
		są zmienne stochastyczne nie podlegające modelowaniu
		(zmienne egzogeniczne): \[
			\boldsymbol{Y}_{t} =
				\sum_{i=1}^{p}
					\boldsymbol{A}_{i} \boldsymbol{Y}_{t-i} +
				\boldsymbol{B} \boldsymbol{Z}_t +
				\boldsymbol{\epsilon}_{t},
		\] gdzie
		\begin{itemize}
			\item $\boldsymbol{Z}_t$ --- macierz zmiennych
			egzogenicznych,
			\item $\boldsymbol{B}$ --- macierz parametrów
				związanych z $\boldsymbol{Z}_t.$
		\end{itemize}
	\end{frame}
	
	\begin{frame}{Egzogeniczność}
		\framesubtitle{Definicje}
		
		\begin{block}{\textbf{Słaba egzogeniczność}}
			Zmienna $z_t$ jest nazywana słabo egzogeniczną dla wektora parametrów
			$\boldsymbol{\theta}$ jeżeli estymacja parametrów $\boldsymbol{\theta}$
			z modelu warunkowego warunkowanego zmienną $z_t$ nie powoduje utraty
			informacji względem estymacji tych parametrów z modelu pełnego.
		\end{block}
		
		\begin{block}{\textbf{Silna egzogeniczność}}
			Zmienna $z_t$ jest nazywana silnie egzogeniczną dla wektora parametrów
			$\boldsymbol{\theta},$ jeżeli jest ona dla niego słabo egzogeniczna oraz
			prognoza $\boldsymbol{Y}_t$ może zostać przeprowadzona bez utraty
			dokładności na podstawie modelu warunkowego pod warunkiem $z_t.$
		\end{block}
		
	\end{frame}

	\begin{frame}{Selekcja optymalnego $p$}
		Metody selekcji optymalnej wartości parametru $p:$
		\begin{itemize}
			\item przesłanki teoretyczne,
			\item kryteria informacyjne,
			\item testy istotności parametrów dla ostatnich opóźnień,
			\item analiza reszt modelu (np. statystyka Ljunga-Boxa).
		\end{itemize}
	\end{frame}
	
	\begin{frame}{Selekcja optymalnego $p$}
		\framesubtitle{Kryteria informacyjne}
		Wyboru optymalnej wartości parametru $p$ dokonuje się 
		minimalizując wartość wybranego kryterium informacyjnego.
		\begin{block}{\textbf{Kryterium Akaike}}
			\[
				AIC =
					-2\frac{\ln\left(L\right)}{T} + 
					k\frac{2}{T}
			\]
		\end{block}
		\begin{block}{\textbf{Kryterium Schwarza}}
			\[
				SC =
					-2\frac{\ln\left(L\right)}{T} +
					k \frac{\ln\left(T\right)}{T}
			\]
		\end{block}
		\begin{block}{\textbf{Kryterium Hannana-Quinna}}
			\[
				HQ = 
					-2\frac{\ln\left(L\right)}{T} +
					k \frac{\ln\left(\ln\left(T\right)\right)}{T}
			\]
		\end{block}
	\end{frame}

	\begin{frame}{Estymacja}
		Jeżeli każde z równań modelu VAR posiada ten sam zbiór zmiennych
		objaśniających, to estymacja parametrów całego modelu może zostać
		przeprowadzona poprzez zastosowanie metody najmniejszych kwadratów
		do każdego z równań osobno.
		\\~\\
		Alternatywą jest zastosowanie metody największej wiarogodności do
		wszystkich równań modelu łącznie.
	\end{frame}
	
	\section{Kopula}
	
	\begin{frame}{Definicja}
		\begin{block}{\textbf{Kopula}}
			Funkcja $C:\mathbb{I}^{2}\rightarrow\mathbb{I}$ spełniająca następujące warunki
			\begin{enumerate}
				\item $C\left(0,t\right) = C\left(t,0\right) = 0,$
        \item $C\left(1,t\right) = C\left(t,1\right) = t,$
				\item $C\left(u_2, v_2\right) - C\left(u_1, v_2\right) - C\left(u_2, v_1\right) + C\left(u_1 ,v_1\right) \geq 0,$
			\end{enumerate}
			dla $t, u_1, u_2, v_1, v_2 \in \mathbb{I},$ $u_1 \leq u_2,$ $v_1 \leq v_2,$
		\end{block}
    
		\begin{alert}{\textbf{Intuicja.}}
			Kopula to dystrybuanta wielowymiarowa, której dystrybuanty brzegowe pochodzą z rozkładu jednostajnego na $\mathbb{I}$.
		\end{alert}
	\end{frame}
	
	\begin{frame}{Twierdzenie Sklara}

		\begin{block}{\textbf{Twierdzenie}}
			\textbf{Niech:}
			\begin{itemize}
				\item $F$ --- dystrybuanta łączna 2-wymiarowa,
				\item $F_{1}, F_{2}$ --- dystrybuanty brzegowe $F.$
			\end{itemize}
			\textbf{Wówczas:}
			\begin{itemize}
				\item istnieje kopula $C$ taka, że dla dowolnego punktu
					$\left(x_{1}, x_{2}\right) \in \overline{\mathbb{R}}^{2}$
					zachodzi
					\begin{equation}\label{eq:sklartheorem}
						F\left(x_{1}, x_{2}\right) = C\left(F_1\left(x_1\right), F_2\left(x_2\right)\right).
					\end{equation}
			\end{itemize}
			\textbf{Ponadto:}
			\begin{itemize}
				\item jeśli $F_{1}, F_{2}$ są ciągłe, to $C$ jest wyznaczona jednoznacznie.
				\item jeśli funkcja $C$ jest kopulą, a  $F_{1}, F_{2}$ są dystrybuantami, to zależność
					\eqref{eq:sklartheorem} wyznacza dystrybuantę wielowymiarową $F.$
			\end{itemize}
		\end{block}
		
	\end{frame}
	
	\begin{frame}{Twierdzenie Sklara}
		\framesubtitle{Wnioski}
		
		\begin{alert}{\textbf{Interpretacja.}}
			Problem modelowania wielowymiarowego szeregu czasowego
			rozdzielić można na dwa podproblemy
			\begin{itemize}
				\item modelowania jednowymiarowych szeregów składowych,
				\item modelowania współzależności łączących te szeregi.
			\end{itemize}
		\end{alert}
		
		\begin{alert}{\textbf{Uwaga.}}
			Twierdzenie Sklara przenosi się bez zmian również na
			przypadek rozkładów warunkowych.
		\end{alert}
		
		\begin{alert}{\textbf{Uwaga.}}
			Łączenie jednowymiarowych szeregów czasowych za pomocą różnych
			kopul pozwala skupić się na modelowaniu różnych współzależności.
		\end{alert}
	\end{frame}
	
	\begin{frame}{Miary współzależności}
		\framesubtitle{Współczynnik Kendalla}
		\begin{block}{\textbf{Współczynnik $\tau$ Kendalla}}
			Niech $X_{i,1}, X_{i,2} \sim X_{i}$ dla $i=1,2.$ Wówczas współczynnikiem 
			$\tau$ Kendalla zmiennych $X_1$ i $X_2$ nazwiemy \begin{eqnarray*}
					\tau\left(X_{1},X_{2}\right) & = &
						P\left(\left(X_{1,1}-X_{1,2}\right)\left(X_{2,1}-X_{2,2}\right)>0\right) - \\
						& & P\left(\left(X_{1,1}-X_{1,2}\right)\left(X_{2,1}-X_{2,2}\right)<0\right).
				\end{eqnarray*}
		\end{block}
  
		\begin{block}{\textbf{Twierdzenie}}
			Niech $X_{1}$ i $X_{2}$ będą ciągłymi zmiennymi losowymi o współzależnościach opisanych kopulą C.
			Wówczas $\tau$ Kendalla zmiennych losowych $X_{1}$ i $X_{2}$ daje się przedstawić jako \[
				\tau\left(X_{1},X_{2}\right)=4\int_{0}^{1}\int_{0}^{1}C\left(u,v\right)dC\left(u,v\right)-1.
			\]
		\end{block}
	\end{frame}
	
	\begin{frame}{Miary współzależności}
		\framesubtitle{Współczynnik Spearmana}
		\begin{block}{\textbf{Współczynnik $\rho$ Spearmana}}
			Niech $X_1$ i $X_2$ będą ciągłymi zmiennymi losowymi o dystrybuantach odpowiednio
			$F_1$ i $F_2.$ Niech $\rho$ oznacza współczynnik korelacji Pearsona. Wówczas współczynnikiem
			$\rho$ Spearmana nazwiemy \[
				\rho_{S}\left(X_{1},X_{2}\right)=\rho\left(F_{1}\left(X_{1}\right),F_{2}\left(X_{2}\right)\right).
			\]
		\end{block}
	
		\begin{block}{\textbf{Twierdzenie}}
			Niech $X_{1}$ i $X_{2}$ będą ciągłymi zmiennymi losowymi o współzależnościach opisanych kopulą C.
			Wówczas $\rho$ Spearmana zmiennych losowych $X_{1}$ i $X_{2}$ daje się przedstawić jako \[
				\rho_{S}\left(X_{1},X_{2}\right)=12\int_{0}^{1}\int_{0}^{1}C\left(u,v\right)dudv-3
			\]
		\end{block}
	\end{frame}
	
	\begin{frame}{Miary współzależności}
		\framesubtitle{Współczynniki zależności asymptotycznych}
		\begin{block}{\textbf{Współczynnik zależności w dolnym ogonie}}
			Niech $X_{1}$ i $X_{2}$ będą ciągłymi zmiennymi losowymi o dystrybuantach
			odpowiednio $F_{1}$ i $F_{2}.$ Wówczas współczynnikiem zależności w
			dolnym ogonie nazwiemy granicę \[
				\lambda_{L} = 
					\lim_{\alpha\rightarrow0+} P\left(
						X_{2}\leq F_{2}^{-1}\left(\alpha\right)\mid X_{1}\leq F_{1}^{-1}\left(\alpha\right)
					\right)
			\] o ile granica ta istnieje.
		\end{block}
	\end{frame}
	
	\begin{frame}{Miary współzależności}
		\framesubtitle{Współczynniki zależności asymptotycznych}
		\begin{block}{\textbf{Współczynnik zależności w górnym ogonie}}
			Niech $X_{1}$ i $X_{2}$ będą ciągłymi zmiennymi losowymi o dystrybuantach
			odpowiednio $F_{1}$ i $F_{2}.$ Wówczas współczynnikiem zależności w
			górnym ogonie nazwiemy granicę \[
				\lambda_{U} = 
					\lim_{\alpha\rightarrow1-}P\left(
						X_{2}>F_{2}^{-1}\left(\alpha\right)\mid X_{1}>F_{1}^{-1}\left(\alpha\right)
					\right)
			\] o ile granica ta istnieje.
		\end{block}
	\end{frame}
	
	\begin{frame}{Miary współzależności}
		\framesubtitle{Współczynniki zależności asymptotycznych}
		\begin{block}{\textbf{Twierdzenie}}
			Niech $X_{1}$ i $X_{2}$ będą ciągłymi zmiennymi losowymi o współzależnościach opisanych kopulą C.
			Wówczas współczynniki zależności asymptotycznych zmiennych losowych $X_{1}$ i $X_{2}$ dają się przedstawić jako \begin{eqnarray*}
				\lambda_{L}\left(X_{1},X_{2}\right) & = & \lim_{\alpha\rightarrow0+}\frac{C\left(\alpha,\alpha\right)}{\alpha},\\
				\lambda_{U}\left(X_{1},X_{2}\right) & = & 2-\lim_{\alpha\rightarrow1-}\frac{1-C\left(\alpha,\alpha\right)}{1-\alpha}.
			\end{eqnarray*}
		\end{block}
	\end{frame}
	
	\begin{frame}{Estymacja}
		\framesubtitle{Metoda IFM}
		Zastosowanie twierdzenia Sklara pozwala przedstawić funkcję log-wiarogodności
		wielowymiarowego szeregu czasowego jako: \[
			\ell\left(\boldsymbol{\theta}, \boldsymbol{\alpha}_1, \boldsymbol{\alpha}_2\right) =
				\sum_{i=1}^{2} \ell_{i}\left(\boldsymbol{\alpha}_i\right) +
				\ell_C\left(\boldsymbol{\theta}, \boldsymbol{\alpha}_1, \boldsymbol{\alpha}_2\right),
		\] gdzie
		\begin{itemize}
			\item $\ell_i$ --- funkcja log-wiarogodności i-tego jednowymiarowego szeregu czasowego brzegowego,
			\item $\ell_C$ --- funkcja log-wiarogodności kopuli,
			\item $\boldsymbol{\alpha}_i$ --- wektor parametrów i-tego jednowymiarowego szeregu czasowego brzegowego,
			\item $\boldsymbol{\theta}$ --- wektor parametrów kopuli.
		\end{itemize}
	\end{frame}
	
	\begin{frame}{Estymacja}
		\framesubtitle{Metoda IFM}
		
		Przedstawiona postać funkcji $\ell\left(\boldsymbol{\theta}, \boldsymbol{\alpha}_1, \boldsymbol{\alpha}_2\right)$
		sugeruje możliwość rozdzielenia procesu estymacji parametrów modelu metodą największej wiarogodności na dwa etapy:
		\begin{enumerate}
			\item estymację parametrów jednowymiarowych szeregów czasowych $\boldsymbol{\alpha}_1$ i $\boldsymbol{\alpha}_2$
				na podstawie odpowiednio $\ell_1\left(\boldsymbol{\alpha}_1\right)$ i $\ell_2\left(\boldsymbol{\alpha}_2\right),$
			\item estymację parametrów kopuli $\boldsymbol{\theta}$ na podstawie
				$\ell_C\left(\boldsymbol{\theta}, \hat{\boldsymbol{\alpha}}_1, \hat{\boldsymbol{\alpha}}_2\right),$ gdzie
				$\hat{\boldsymbol{\alpha}}_1$ i $\hat{\boldsymbol{\alpha}}_2$ to estymatory parametrów odpowiednio
				$\boldsymbol{\alpha}_1$ i $\boldsymbol{\alpha}_2$ uzyskane w pierwszym kroku estymacji.
		\end{enumerate}
		
		Powyższa metoda estymacji nosi nozwę \textbf{metody IFM}.
	\end{frame}
	
	\begin{frame}{Estymacja}
		\framesubtitle{Metoda IFM}
		\begin{alert}{\textbf{Uwaga}}
			Zastało udowodnione, że metoda IFM posiada satysfakcjonujące właściwości statystyczne.
		\end{alert}
		\\~\\
		\begin{alert}{\textbf{Uwaga.}}
			Metoda IFM pozwala stanowczo uprościć proces estymacji parametrów złożonego modelu wielowymiarowego
			poprzez rozłożenie go na dwa o wiele prostsze podproblemy, estymacji parametrów jednowymiarowych modeli
			brzegowych oraz estymacji parametrów opisujących współzależności tych modeli.
		\end{alert}
	\end{frame}
	
	\begin{frame}{Przykład}
		\framesubtitle{Kopule eliptyczne}
		
		\begin{block}{\textbf{Kopula normalna}}
			Kopulą normalną o korelacji liniowej $\rho$ nazwiemy funkcję \[
				C_{\rho}^{g}\left(u_{1},u_{2}\right) = 
					\frac{1}{2\pi\sqrt{1-\rho^{2}}} \int_{-\infty}^{z_{u_{1}}} \int_{-\infty}^{z_{u_{2}}}
						e^{\frac{2\rho s_{1}s_{2}-s_{1}^{2}-s_{2}^{2}}{2\left(1-\rho^{2}\right)}}
					ds_{1}ds_{2},
			\] gdzie
			\begin{itemize}
				\item $z_{u_i}$ --- kwantyl $u_i$ standardowego rozkładu normalnego.
			\end{itemize}
		\end{block}
	\end{frame}
	
	\begin{frame}{Przykład}
		\framesubtitle{Kopule eliptyczne}
		
		\begin{block}{\textbf{Kopula t-Studenta}}
			Kopulą t-Studenta o $\nu$ stopniach swobody i korelacji liniowej $\rho$ nazwiemy funkcję \[
				C_{\nu,\rho}^{t}\left(u_{1},u_{2}\right) =\]\[
					\frac{1}{2\pi\sqrt{1-\rho^{2}}} \int_{-\infty}^{t_{\nu}^{-1}\left(u_{1}\right)}\int_{-\infty}^{t_{\nu}^{-1}\left(u_{2}\right)}
						\left(1+\frac{s_{1}^{2}+s_{2}^{2}-2\rho s_{1}s_{2}}{\nu\left(1-\rho^{2}\right)}\right)^{-\frac{\nu+2}{2}}
					ds_{1}ds_{2},
			\] gdzie
			\begin{itemize}
				\item $t_{\nu}^{-1}\left(u_{i}\right)$ --- kwantyl $u_i$ rozkładu t-Studenta o $\nu$ st. swobody.
				\item $\Gamma\left(\alpha\right)=\int_{0}^{\infty}x^{\alpha-1}e^{-x}dx,\mbox{ dla }\alpha>0.$
			\end{itemize}
		\end{block}
	\end{frame}
	
	\begin{frame}{Przykład}
		\framesubtitle{Model Copula-GARCH}
		
		\begin{block}{\textbf{Model Copula-GARCH}}
			Niech $r_{1,t}, r_{2,t} \sim \mbox{GARCH}.$ Niech współzależności $r_{1,t}$ i $r_{2,t}$
			opisuje kopula C. Modelem Copula-GARCH szeregów czasowych $r_{1,t}$ i $r_{2,t}$ nazwiemy
			model opisany dystrybuantą wielowymiarową postaci \[
				F\left(r_{1,t}, r_{2,t}\mid \boldsymbol{{\cal R}}_{t-1}\right) = C\left(
					F_1\left(r_{1,t}\mid \boldsymbol{{\cal R}}_{t-1}\right),
					F_2\left(r_{2,t}\mid \boldsymbol{{\cal R}}_{t-1}\right)
					\mid
					\boldsymbol{{\cal R}}_{t-1}
				\right),
			\] gdzie
			\begin{itemize}
				\item $F_i$ --- dystrybuanta $r_{i,t}$ dla $i=1,2,$
				\item $F$ --- dystrybuanta łączna $r_{1,t}$ i $r_{2,t}.$
			\end{itemize}
		\end{block}
		
	\end{frame}

	\section*{}

	\begin{frame}
		\center
		\Huge \bfseries
		Pytania?
	\end{frame}

	\begin{frame}
		\center
		\Huge \bfseries
		Dziękuję za uwagę!
	\end{frame}

\end{document}
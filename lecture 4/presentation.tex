\documentclass[a4paper, 11pt]{beamer}

\usepackage{polski}
\usepackage[utf8]{inputenc}
\usepackage{units}

\mode<presentation> {
	\usetheme{Frankfurt}
	\setbeamercovered{transparent}
	\usecolortheme{default}
}

\title{Ekonometria Finansowa}
\subtitle{Nieliniowe modele szeregów czasowych}
\author{mgr Paweł Jamer\thanks{pawel.jamer@gmail.com}}

\begin{document}

	\begin{frame}
		\titlepage
	\end{frame}
	
	\section{Wprowadzenie}
	
	\begin{frame}{Problem}
		\begin{block}{\textbf{Problem}}
			Poszukiwany jest model opisujący szereg czasowy $r_t$ równaniem postaci \[
				r_t = f\left(\epsilon_{t}, \epsilon_{t-1}, \epsilon_{t-2}, \ldots\right),
			\] gdzie
			\begin{itemize}
				\item $f$ --- dowolna funkcja nieliniowa,
				\item $\epsilon_{t}$ --- $\mbox{WN}\left(0,1\right)$
			\end{itemize}
		\end{block}
	\end{frame}
	
	\begin{frame}{Uproszczenie problemu}
		Uprośćmy przedstawiony problem rozwijając funkcję $f$ w szereg Taylora: \[
			r_t = g\left(\epsilon_{t-1}, \epsilon_{t-2}, \ldots\right) + \epsilon_t h\left(\epsilon_{t-1}, \epsilon_{t-2}, \ldots\right).
		\]
		Wówczas funkcje $g$ oraz $h$ możemy interpretować następująco:
		\begin{itemize}
			\item $g\left(\epsilon_{t-1}, \epsilon_{t-2}, \ldots\right) = \mathbb{E}\left(r_t \mid \mathcal{R}_{t-1}\right),$
			\item $h^2\left(\epsilon_{t-1}, \epsilon_{t-2}, \ldots\right) = \mbox{Var}\left(r_t \mid \mathcal{R}_{t-1}\right).$
		\end{itemize}
	\end{frame}
	
	\begin{frame}{Definicja}
		\begin{block}{\textbf{Model nieliniowy}}
			Model opisujący proces $r_t$ nazwiemy nieliniowym, jeżeli
			warunkowa wartość oczekiwana $\mathbb{E}\left(r_t \mid \mathcal{R}_{t-1}\right)$ lub
			warunkowa wariancja $\mbox{Var}\left(r_t \mid \mathcal{R}_{t-1}\right)$ jest
			w nim opisana za pomocą funkcji nieliniowej.
		\end{block}
	\end{frame}
	
	\section{Przegląd modeli}
	
	\begin{frame}{Model NLAR}
		\framesubtitle{Definicja}
		\begin{block}{\textbf{Model nieliniowej autoregresji}}
			Modelem nieliniowej autoregresji rzędu $p$ szeregu czasowego $r_t$ nazwiemy model
			opisany równaniem \[
				y_t = f\left(y_{t-1}, y_{t-2}, \ldots, y_{t-p}\right) + \epsilon_t.
			\]
		\end{block}
		\begin{alert}{\textbf{Oznaczenie.}}
			Model nieliniowej autoregresji rzędu $p$ oznacza się symbolem $\mbox{NLAR}\left(p\right).$
		\end{alert}
	\end{frame}
	
	\begin{frame}{Model NLAR}
		\framesubtitle{Przykład}
		\begin{block}{\textbf{Model wykładniczej autoregresji}}
			Modelem wykładniczej autoregresji rzędu $p$ szeregu czasowego $r_t$ nazwiemy model
			opisany równaniem \[
				y_t = \sum_{j=1}^{p} \alpha_j y_{t-j} + \epsilon_t,
			\] gdzie
			\begin{itemize}
				\item $\alpha_j = \theta_j + \pi_j e^{-\alpha y_{t-j}^2}.$
			\end{itemize}
		\end{block}
		\begin{alert}{\textbf{Oznaczenie.}}
			Model wykładniczej autoregresji rzędu $p$ oznacza się symbolem $\mbox{EAR}\left(p\right).$
		\end{alert}
	\end{frame}
	
	\begin{frame}{Model NLMA}
		\framesubtitle{Definicja}
		\begin{block}{\textbf{Model nieliniowej średniej ruchomej}}
			Modelem nieliniowej średniej ruchomej rzędu $q$ szeregu czasowego $r_t$ nazwiemy model
			opisany równaniem \[
				y_t = f\left(\epsilon_{t-1}, \epsilon_{t-2}, \ldots, \epsilon_{t-q}\right) + \epsilon_t.
			\]
		\end{block}
		\begin{alert}{\textbf{Oznaczenie.}}
			Model nieliniowej średniej ruchomej rzędu $q$ oznacza się symbolem $\mbox{NLMA}\left(q\right).$
		\end{alert}
	\end{frame}
	
	\begin{frame}{Model NLMA}
		\framesubtitle{Przykład}
		\begin{block}{\textbf{Model Rocke'a}}
			Modelem Rocke'a szeregu czasowego $r_t$ nazwiemy model opisany równaniem \[
				y_t = \epsilon_t + \sum_{j=1}^{\infty} b_j \psi\left(\epsilon_{t-j}\right),
			\] gdzie
			\begin{itemize}
				\item $\psi\left(x\right) \leq 0$ dla $x > 0,$
				\item $\psi\left(-x\right)=\psi\left(x\right),$
				\item $\psi\left(0\right) = 1.$
			\end{itemize}
		\end{block}
	\end{frame}
	
	\begin{frame}{Model dwuliniowy}
		\framesubtitle{Definicja}
		\begin{block}{\textbf{Model dwuliniowy}}
			Modelem dwuliniowym szeregu czasowego $r_t$ nazwiemy model
			opisany równaniem \[
				y_t = \mu + 
					\sum_{i=1}^{p} \alpha_i y_{t-i} + 
					\sum_{j=0}^{q} \theta_j \epsilon_{t-j} + 
					\sum_{k=1}^{P} \sum_{l=1}^{Q} c_{k,l} y_{t-k} \epsilon_{t-l},
			\] gdzie
			\begin{itemize}
				\item $\theta_0 = 1.$
			\end{itemize}
		\end{block}
	\end{frame}
	
	\begin{frame}{Model dwuliniowy}
		\framesubtitle{Właściwości}
		\begin{alert}{\textbf{Uwaga.}}
			Model dwuliniowy uwzględnia skupianie się danyh.
		\end{alert}
		\\~\\
		\begin{alert}{\textbf{Uwaga.}}
			$r_t \sim \mbox{ARCH} \Rightarrow r_t^2$ - proces dwuliniowy.
		\end{alert}
		\\~\\
		\textbf{Wśród modeli dwuliniowych wyróżnia się:}
		\begin{itemize}
			\item modele dwuliniowe naddiagonalne ($c_{k,l} = 0$ dla $k > l$).
			\item modele dwuliniowe poddiagonalne ($c_{k,l} = 0$ dla $k < l$).
			\item modele dwuliniowe diagonalne ($c_{k,l} = 0$ dla $k \neq l$).
		\end{itemize}
	\end{frame}
	
	\begin{frame}{Model TAR}
		\begin{block}{\textbf{Model progowy autoregresyjny}}
			Modelem progowym autoregresyjnym (TAR) szeregu czasowego $r_t$ nazwiemy model opisany równaniem \[
				r_t + \theta_0 + \sum_{i=1}^p \theta_i r_{t-i} + 
					I\left(z_{t-d} \leq r\right) \left(\psi_0 + \sum_{j=1}^{q} \psi_j r_{t-j}\right) = \epsilon_t.
			\]
		\end{block}
		
		\begin{alert}{\textbf{Model SETAR.}}
			Model TAR w którym $z_{t-d} = r_{t-d}$ nazwiemy modelem SETAR.
		\end{alert}
	\end{frame}
	
	\begin{frame}{Model STR}
		\framesubtitle{Definicja}
		\begin{block}{\textbf{Model wygładzonego przejścia}}
			Modelem wygładzonego przejścia szeregu czasowego $r_t$ nazwiemy model opisany równaniem \[
				r_t = 
					\boldsymbol{x}_{t}^{\prime} \boldsymbol{\pi}_{1} + 
					\boldsymbol{x}_{t}^{\prime} \boldsymbol{\pi}_{2} F\left(z_t\right) + 
					\boldsymbol{\epsilon}_{t},
			\] gdzie
			\begin{itemize}
				\item $\boldsymbol{x}_{t} = \left[\begin{matrix}
						1 & y_{t-1} & \cdots & y_{t-p} & x_{1,t} & \cdots & x_{q,t}
					\end{matrix}\right]$ --- wektor zmiennych,
				\item $\boldsymbol{\pi}_{1}, \boldsymbol{\pi}_{2}$ --- wektory parametrów,
				\item $\epsilon_t$ --- WN,
				\item $F\left(z_t\right)$ --- funkcja transformacji, parzysta lub nieparzysta funkcja ciągła.
			\end{itemize}
		\end{block}
	\end{frame}
	
	\begin{frame}{Model STR}
		\framesubtitle{Postać funkcji transformacji}
		\textbf{Funkcja logistyczna:} \[
			F\left(z_t\right) = F\left(t\right) = \frac{1}{
				1 + e^{-\gamma\left(t^k + \sum_{i=1}^{k} \alpha_i t^{k-i}\right)}
			}.
		\]
		\textbf{Funkcja wykładnicza:} \[
			F\left(z_t\right) = F\left(t\right) = 
				1 + e^{-\gamma\left(t - \alpha\right)^2}.
		\]
	\end{frame}
	
	\begin{frame}{Model przełącznikowy}
		\begin{block}{\textbf{Model przełącznikowy}}
			Modelem przełącznikowym szeregu czasowego $r_t$ nazwiemy model opisany równaniem \[
				r_t = \alpha_{z_t} + \sum_{i=1}^{p} \beta_{z_t,i} r_{t-i} + \epsilon_t,
			\] gdzie
			\begin{itemize}
				\item $z_{t}$ --- łańcuch Markowa o zbiorze stanów \[\mathcal{S} = \left\{1, 2, \ldots, k\right\}, k \geq 2.\]
			\end{itemize}
		\end{block}
	\end{frame}
	
	\section{Wybór modelu}
	
	\begin{frame}{Wybór funkcji nieliniowej}
		\framesubtitle{Podstawy teoretyczne}
		\textbf{Odpowiednio dobrana funkcja nieliniowa powinna:}
		\begin{itemize}
			\item zachowywać zgodność ze znanymi zależnościami ekonomicznymi,
			\item zachowywać zgodność ze znanymi faktami empirycznymi,
			\item posiadać dziedzinę zmienności zgodną z faktyczną zmiennością modelowanej wielkości,
			\item charakteryzować się elastycznością umożliwiającą aproksymację form zbliżonych do analizowanej formy.
			\item umożliwiać łatwe wyznaczenie jej parametrów.
		\end{itemize}
	\end{frame}
	
	\begin{frame}{Wybór funkcji nieliniowej}
		\framesubtitle{Funkcja korelacji}
		\textbf{Współczynnik maksymalnej korelacji:} \[
			m_{\rho} = \max_{f,g} \mbox{Cor}\left(g\left(y\right), f\left(x\right)\right).
		\]
		\textbf{Maksymalna średnia korelacja:} \[
			m_{m} = \max_{f} \mbox{Cor}\left(y, f\left(x\right)\right).
		\]
		\textbf{Maksymalny współczynnik regresji} \[
			m_{r} = \max_{f} R^2,
		\] gdzie
		\begin{itemize}
			\item $R^2$ --- współczynnik determinacji w regresji $y=f\left(x\right)+\epsilon_t.$
		\end{itemize}
	\end{frame}
	
	\begin{frame}{Wybór liczby opóźnień}
		\begin{alert}{\textbf{Uwaga.}}
			Wyboru liczby opóźnień w modelach NLAR, NLMA, dwuliniowych oraz progowych można dokonać
			wykorzystując kryteria informacyjne. Najczęściej stosowanymi kryteriami są kryteria AIC oraz BIC.
		\end{alert}
	\end{frame}
	
	\section{Testowanie nieliniowości}
	
	\begin{frame}{Test RESET}
		\framesubtitle{Algorytm}
		\begin{enumerate}
			\item Stosujemy MNK w celu wyznaczenia parametrów modelu \[
					y_t = \boldsymbol{\beta}^{\prime} \boldsymbol{z}_t + u_t,
				\] gdzie $\boldsymbol{z}_{t} = \left[\begin{matrix}1 & y_{t-1} & \cdots & y_{t-p} & x_{1} & \cdots & x_{k}\end{matrix}\right].$
			\item Obliczamy \[
				\mbox{SSR}_0 = \sum_{t=1}^{T} \hat{u}_t^2.
			\]
			\item Stosujemy MNK w celu wyznaczenia parametrów modelu \[
					\hat{u}_t = \boldsymbol{\delta}^{\prime} \boldsymbol{z}_t + \sum_{j=2}^{h} \psi_j \hat{y}_t^j + v_t.
				\]
			\item Obliczamy \[
				\mbox{SSR} = \sum_{t=1}^{T} \hat{v}_t^2.
			\]
		\end{enumerate}
	\end{frame}
	
	\begin{frame}{Test RESET}
		\framesubtitle{Metoda testowania}
		Testujemy hipotezę \[
			\begin{cases}
				H_{0}: & \left(\forall j\right)\psi_{j}=0,\\
				H_{1}: & \left(\exists j\right)\psi_{j}\neq0.
			\end{cases}
		\]
		Statystyka testowa jest postaci: \[
			F=\frac{
				\nicefrac{\left(\mbox{SSR}_0 - \mbox{SSR}\right)}{\left(h-1\right)}
			}{
				\nicefrac{\left(\mbox{SSR}\right)}{\left(T-m-h\right)}
			} \sim \mathbb{F}^{\left[h-1, T-m-h\right]},
		\] gdzie $m=p+k.$
	\end{frame}
	
	\section*{}
	
	\begin{frame}
		\center
		\Huge \bfseries
		Pytania?
	\end{frame}

	\begin{frame}
		\center
		\Huge \bfseries
		Dziękuję za uwagę!
	\end{frame}

\end{document}
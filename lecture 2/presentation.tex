\documentclass[a4paper, 11pt]{beamer}

\usepackage{polski}
\usepackage[utf8]{inputenc}
\usepackage[utf8]{inputenc}

\mode<presentation> {
	\usetheme{Frankfurt}
	\setbeamercovered{transparent}
	\usecolortheme{default}
}

\title{Ekonometria Finansowa}
\subtitle{Jednowymiarowe modele szeregów czasowych}
\author{mgr Paweł Jamer\thanks{pawel.jamer@gmail.com}}

\begin{document}

	\begin{frame}
		\titlepage
	\end{frame}
	
	\section{Biały szum}
	
	\begin{frame}{Biały szum}
		\begin{block}{\textbf{Biały szum}}
			Białym szumem nazwiemy szereg czasowy $\epsilon_t$ niezależnych zmiennych losowych o tym samym rozkładzie taki, że \begin{eqnarray*}
				\mathbb{E}\left(\epsilon_t\right) & = & 0,\\
				\mbox{Var}\left(\epsilon_t\right) & = & \sigma^2.
			\end{eqnarray*} Biały szum oznaczać będziemy symbolem $\mbox{WN}\left(0, \sigma^2\right)$.
		\end{block}
		\begin{alert}{\textbf{Uwaga}}
			Bardziej złożone modele szeregów czasowych wykorzystują biały szum do opisu niepewności pomiaru opisywanych przez nie wielkości.
		\end{alert}
	\end{frame}
	
	\section{Błądzenie losowe}
	
	\begin{frame}{Błądzenie losowe}
		\begin{block}{\textbf{Błądzenie losowe (bez dryftu)}}
			Szereg czasowy $p_t$ nazwiemy błądzeniem losowym bez dryftu, jeżeli spełnia on równanie \[
				p_t = p_{t-1} + \epsilon_t,
			\] gdzie
			\begin{itemize}
				\item $\epsilon_t$ --- biały szum.
			\end{itemize}
		\end{block}
		\begin{alert}{\textbf{Uwaga.}}
			Uzupełniając powyższy wzór o niezerową stałą $\alpha$ \[
				p_t = \alpha + p_{t-1} + \epsilon_t
			\] uzyskujemy proces błądzenia losowego z dryftem.
		\end{alert}
	\end{frame}
	
	\begin{frame}{Ceny instrumentów finansowych}
		\begin{block}{\textbf{Hipoteza}}
			Cena instrumentu finansowego $p_t$ jest błądzeniem losowym.
		\end{block}
		Rozważmy model \[
			p_t = \alpha + \rho p_{t-1} + \epsilon_t.
		\]
		Prawdziwość powyższej hipotezy jest równoznaczna z tym, że:
		\begin{itemize}
			\item $\hat{\rho}$ statystycznie nie różni się od jedności,
			\item $\epsilon_t$ jest białym szumem.
		\end{itemize}
		Ponadto, jeżeli na zadanym poziomie istotności zachodzi:
		\begin{itemize}
			\item $\hat{\alpha} = 0,$ to $p_t$ jest błądzeniem losowym bez dryfu,
			\item $\hat{\alpha} \neq 0,$ to $p_t$ jest błądzeniem losowym z dryfem.
		\end{itemize}
		\begin{alert}{\textbf{Uwaga.}}
			Z powodu możliwej niestacjonarności $p_t$ estymacja powyższego równania
			jest problematyczna.
		\end{alert}
	\end{frame}
	
	\begin{frame}{Właściwości błądzenia losowego}
		\begin{columns}[onlytextwidth]
			\begin{column}{0.5\textwidth}
				\textbf{Błądzenie losowe bez dryftu} \[
					p_t = p_{t-1} + \epsilon_t,
				\] \[
					p_t = p_0 + \sum_{h=0}^{t} \epsilon_{t-h},
				\] \[
					\mathbb{E}\left(p_t\right) = p_0,
				\] \[
					\mbox{Var}\left(p_t\right) = t \sigma^2_{\epsilon_t}.
				\]
			\end{column}
			\begin{column}{0.5\textwidth}
				\textbf{Błądzenie losowe z dryftem} \[
					p_t = \alpha + p_{t-1} + \epsilon_t,
				\] \[
					p_t = p_0 + t \alpha + \sum_{h=0}^{t} \epsilon_{t-h},
				\] \[
					\mathbb{E}\left(p_t\right) = p_0 + t \alpha,
				\] \[
					\mbox{Var}\left(p_t\right) = t \sigma^2_{\epsilon_t}.
				\]
			\end{column}
		\end{columns}
	\end{frame}
	
	\begin{frame}{Stopy zwrotu instrumentów finansowych}
		Rozważmy model błądzenia losowego bez dryftu dla logarytmu cen pewnego instrumentu finansowego \[
			\log\left(p_t\right) = \log\left(p_{t-1}\right) + \epsilon_t.
		\] Model ten przekształcić możemy do postaci \[
			r_t = \log\left(\frac{p_t}{p_{t-1}}\right) = \epsilon_t.
		\]
		\begin{alert}{\textbf{Uwaga.}}
			Badanie czy logarytm cen $p_t$ instrumentu finansowego jest błądzeniem losowym
			sprowadza się do ustalenia, czy logarytmiczne stopy zwrotu $r_t$ tego instrumentu są białym szumem.
		\end{alert}
	\end{frame}
	
	\begin{frame}{Krytyka}
		Optymalna prognoza ceny instrumentu finansowego na okres przyszły, to przyjęcie ceny tego instrumentu z okresu bieżącego.
		\\~\\
		Nie uwzględnia się rentowności zależnej od ryzyka.
	\end{frame}
	
	\section{Modele ARMA}

	\begin{frame}{Proces AR}
		Zdefiniujmy operator \[
			\varphi\left(B\right) = I - \varphi_{1} B - \ldots - \varphi_{p} B^{p},
		\] gdzie $p \in \mathbb{Z}_{+}.$
		\begin{block}{\textbf{Proces AR}}
			Słabo stacjonarny szereg czasowy $X_t$ nazwiemy procesem AR (autoregresyjnym) rzędu $p,$ jeżeli spełnia on równanie \[
				\varphi\left(B\right) X_t = \epsilon_{t},
			\] gdzie $\epsilon_{t} \sim \mbox{WN}\left(0, \sigma^2\right).$
		\end{block}
		\begin{alert}{\textbf{Oznaczenie.}}
			Proces AR rzędu $p$ oznacza się symbolem $\mbox{AR}\left(p\right).$
		\end{alert}
	\end{frame}

	\begin{frame}{Proces MA}
		Zdefiniujmy operator \[
			\theta\left(B\right) = I + \theta_{1} B + \ldots +\theta_{q} B^{q},
		\] gdzie $q \in \mathbb{Z}_{+}.$
		\begin{block}{\textbf{Proces MA}}
			Słabo stacjonarny szereg czasowy $X_t$ nazwiemy procesem MA (średniej ruchomej) rzędu $q,$ jeżeli spełnia on równanie \[
				X_t = \theta\left(B\right) \epsilon_{t},
			\] gdzie $\epsilon_{t} \sim \mbox{WN}\left(0, \sigma^2\right).$
		\end{block}
		\begin{alert}{\textbf{Oznaczenie.}}
			Proces MA rzędu $q$ oznacza się symbolem $\mbox{MA}\left(q\right).$
		\end{alert}
	\end{frame}
	
	\begin{frame}{Proces ARMA}
		\framesubtitle{Definicja}
		\begin{block}{\textbf{Proces ARMA}}
			Słabo stacjonarny szereg czasowy $X_t$ nazwiemy procesem $\mbox{ARMA}\left(p, q\right),$ jeżeli spełnia on równanie \[
				\varphi\left(B\right) X_t = \theta\left(B\right) \epsilon_{t},
			\] gdzie $\epsilon_{t} \sim \mbox{WN}\left(0, \sigma^2\right).$
		\end{block}
	\end{frame}
	
	\begin{frame}{Proces ARMA}
		\framesubtitle{Estymowanie funkcji autokorelacji (ACF)}
		\begin{block}{\textbf{Estymator funkcji autokorelacji}}
			Jako estymator funkcji autokorelacji procesu $X_t$ możemy przyjąć \[
				\hat{\rho}_h = \frac{\sum_{t=1}^{T-h}\left(X_t - \overline{X}\right)\left(X_{t+h} - \overline{X}\right)}{\sum_{t=1}^{T}\left(X_t - \overline{X}\right)^2}
			\]
		\end{block}
	\end{frame}
	
	\begin{frame}{Proces ARMA}
		\framesubtitle{Estymowanie funkcji autokorelacji częściowej (PACF)}
		\begin{block}{\textbf{Estymator funkcji autokorelacji częściowej}}
			Jako estymator funkcji autokorelacji częściowej procesu $X_t$ możemy przyjąć \begin{eqnarray*}
				\alpha_{0} & = & 1,\\
				\alpha_{1} & = & \phi_{1,1} = \hat{\rho}_1,\\
				\alpha_{h} & = & \phi_{h,h} = \frac{
					\hat{\rho}_h - \sum_{j=1}^{h-1} \phi_{h-1,j}{\hat{\rho}_{h-j}}
				}{
					1 - \sum_{j=1}^{h-1} \phi_{h-1,j} \hat{\rho}_j
				}, h = 2, 3, \ldots,
			\end{eqnarray*} gdzie $\phi_{i,j}$ to rozwiązania układu Yule'a-Walkera.
		\end{block}
		\begin{alert}{\textbf{Intuicja.}}
			Współczynnik autokorelacji częściowej mierzy korelację między zmiennymi
			$X_t$ oraz $X_{t-h}$ po eliminacji wpływu zmiennych pośrednich.
		\end{alert}
	\end{frame}
	
	\begin{frame}{Proces ARMA}
		\framesubtitle{Identyfikacja na podstawie ACF oraz PACF}
		\begin{itemize}
			\item Jeżeli proces jest typu $\mbox{AR}\left(p\right),$ to funkcja
				autokorelacji powoli maleje, natomiast funkcja autokorelacji częściowej
				staje się statystycznie równa zero od wartości $p + 1.$
			\item Jeżeli proces jest typu $\mbox{MA}\left(q\right),$ to funkcja
				autokorelacji staje się statystycznie równa zero od wartości $q + 1,$
				natomiast funkcja autokorelacji częściowej powoli maleje.
			\item Jeżeli proces jest typu $\mbox{ARMA}\left(p,q\right),$ to funkcja
				autokorelacji oraz funkcja autokorelacji częściowej łagodnie zanikają.
		\end{itemize}
	\end{frame}
	
	\begin{frame}{Proces ARMA}
		\framesubtitle{Identyfikacja na podstawie kryteriów informacyjnych}
		\begin{block}{\textbf{Idea}}
			Należy oszacować modele $\mbox{ARMA}\left(p,q\right)$ dla wszystkich możliwych
			kombinacji parametrów $p=1,2,\ldots,P$ oraz $q=1,2,\ldots,Q,$ a następnie obliczyć
			dla nich wartości wybranego kryterium informacyjnego. Minimalna wartość kryterium
			wyznacza optymalną parę parametrów $p$ i $q.$
		\end{block}
	\end{frame}
	
	\begin{frame}{Proces ARMA}
		\framesubtitle{Kryteria informacyjne}
		Niech $L_{p,q}$ oznacza wiarogoność modelu $\mbox{ARMA}\left(p, q\right).$
		\begin{block}{\textbf{Kryterium informacyjne Akaike (AIC)}}
			\[
				\mbox{AIC}_{p,q} = -2\ln L_{p,q} + 2\left(p + q + 1\right)
			\]
		\end{block}
		\begin{block}{\textbf{Bayesowskie kryterium informacyjne (BIC)}}
			\[
				\mbox{BIC}_{p,q} = -2\ln L_{p,q} + \left(p + q + 1\right) \ln T
			\]
		\end{block}
	\end{frame}
	
	\begin{frame}{Proces ARMA}
		\framesubtitle{Podział metod estymacji}
		\textbf{Metody estymacji:}
		\begin{itemize}
			\item Metody estymacji parametrów modelu $\mbox{AR}\left(p\right):$
			\begin{itemize}
				\item równania Yule'a-Walkera,
				\item metoda najmniejszych kwadratów,
				\item metoda największej wiarogodności.
			\end{itemize}
			\item Metody estymacji parametrów modelu $\mbox{ARMA}\left(p,q\right):$
			\begin{itemize}
				\item metoda największej wiarogodności.
			\end{itemize}
		\end{itemize}
	\end{frame}
	
	\begin{frame}{Proces ARMA}
		\framesubtitle{Idea metod estymacji}
		\begin{block}{\textbf{Równania Yule'a-Walkera}}
			Równania Yule'a-Walkera to liniowy układ równań pozwalający na wyrażenie
			parametrów autoregresji za pomocą współczynników autokorelacji. Oszacowania
			parametrów modelu AR(p) otrzymujemy zastępując we wspomnianym układzie
			współczynniki autokorelacji ich estymatorami, a następnie rozwiązując układ.
		\end{block}
		\begin{block}{\textbf{Metoda najmniejszych kwadratów}}
			Wykorzystanie klasycznej MNK do stacjonarnego szeregu czasowego typu AR(p).
			Uzysane w ten sposób estymatory będą zgodne z estymatorami metody największej
			wiarogodności.
		\end{block}
		\begin{block}{\textbf{Metoda największej wiarogodności}}
			Metoda polegająca na maksymalizacji funkcji największej wiarogodności procesu
			$\mbox{ARMA}\left(p,q\right).$
		\end{block}
	\end{frame}
	
	\begin{frame}{Proces ARMA}
		\framesubtitle{Prognoza}
		Interesuje nas prognoza na $h$ okresów w przód: \[
			X_{t+h} = \sum_{k=1}^{p} \rho_{k} X_{t+h-k} + \sum_{k=0}^{q} \theta_{k} \epsilon_{t-k},
		\] gdzie $\theta_0 = 1.$
		\\~\\
		Możemy wyrazić ją jako warunkową wartość oczekiwaną: \begin{eqnarray*}
			\hat{X}_t\left(h\right) & = & \mathbb{E}\left(X_{t+h} \mid X_t, X_{t-1}, \ldots\right),\\
			\mathbb{E}\left(X_{t+j} \mid X_t, X_{t-1}, \ldots\right) & = & \begin{cases}
				X_{t+j}, & j \leq 0,\\
				\hat{X}_t\left(j\right), & j > 0,
			\end{cases}\\
			\mathbb{E}\left(\epsilon_{t+j} \mid \epsilon_t, \epsilon_{t-1}, \ldots\right) & = & \begin{cases}
				\epsilon_{t+j}, & j \leq 0,\\
				0, & j > 0.
			\end{cases}
		\end{eqnarray*}
	\end{frame}
	
	\begin{frame}{Proces ARIMA}
		\begin{block}{\textbf{Proces ARIMA}}
			Szereg czasowy $X_t$ nazwiemy procesem $\mbox{ARIMA}\left(p, d, q\right),$ jeżeli
			szereg czasowy $\Delta^{d} X_t$ jest procesem $\mbox{ARMA}\left(p, q\right).$
		\end{block}
		Z powyższej definicji wynika, że proces $\mbox{ARIMA}\left(p, d, q\right)$ jest opisywany przez następujące równanie: \[
			\varphi\left(B\right)\left(\Delta^{d} X_{t}\right) = \theta\left(B\right) \epsilon_{t}.
		\]
	\end{frame}
	
	\begin{frame}{Multiplikatywny proces ARMA}
		\framesubtitle{Intuicja}
		Niech proces $X_t$ charakteryzuje równanie \[
			\varphi_X\left(B\right) X_t = \theta_X\left(B\right) \epsilon_t,
		\] natomiast proces $\epsilon_t$ równanie \[
			\varphi_{\epsilon}\left(B\right) \epsilon_t = \theta_{\epsilon}\left(B\right) \eta_t,
		\] gdzie $\eta_t \sim \mbox{WN}\left(0, \sigma^2\right).$
		Składając powyższe równania uzyskujemy \[
			\varphi_{\epsilon}\left(B\right) \varphi_X\left(B\right) X_t = \theta_X\left(B\right) \theta_{\epsilon}\left(B\right) \eta_t.
		\]
		\begin{alert}{\textbf{Uwaga.}}
			Powyższe złożenie dwóch procesów ARMA pozostaje nadal procesem ARMA.
		\end{alert}
	\end{frame}
	
	\begin{frame}{Multiplikatywny proces ARMA}
		\framesubtitle{Definicja}
		\begin{block}{\textbf{Multiplikatywny proces ARMA}}
			Szereg czasowy $X_t$ nazwiemy multiplikatywnym procesem
			$\mbox{ARMA}\left(p, q\right)\times\left(P,Q\right)_s,$ jeżeli
			spełnia on równanie \[
				\Phi\left(B^s\right) X_t = \Theta\left(B^s\right) \epsilon_t,
			\] gdzie
			\begin{itemize}
				\item $\Phi$ --- operator analogiczny do $\varphi$ rzędu $P,$
				\item $\Theta$ --- operator analogiczny do $\theta$ rzędu $Q,$
				\item $\epsilon_t$ --- proces $\mbox{ARMA}\left(p,q\right).$
			\end{itemize}
		\end{block}
	\end{frame}
	
	\begin{frame}{Multiplikatywny proces ARMA}
		\framesubtitle{Właściwości}
		\begin{alert}{\textbf{Reprezentacja.}}
			Proces $\mbox{ARMA}\left(p, q\right)\times\left(P,Q\right)_s$ jest
			specyficznym przypadkiem procesu $\mbox{ARMA}\left(p+sP,q+sQ\right).$
		\end{alert}
		\\~\\
		\begin{alert}{\textbf{Estymacja.}}
			Estymacja parametrów multiplikatywnych procesów ARMA odbywa się przez osobną
			estymację parametrów każdego z procesów ARMA wchodzących w jego skład.
		\end{alert}
	\end{frame}
	
	\begin{frame}{Sezonowy proces ARIMA}
		\begin{block}{\textbf{Sezonowy proces ARIMA}}
			Szereg czasowy $X_t$ nazwiemy sezonowym procesem
			$\mbox{ARIMA}\left(p, d, q\right)\times\left(P,D,Q\right)_s,$
			jeżeli spełnia on równanie \[
				\Phi\left(B^s\right) \phi\left(B\right) \Delta_s^D \Delta^d X_t = \Theta\left(B^s\right) \theta\left(B\right) \epsilon_t,
			\] gdzie
			\begin{itemize}
				\item $\epsilon_t$ --- biały szum,
				\item $\Delta_s^D \Delta^d X_t$ --- proces stacjonarny.
			\end{itemize}
		\end{block}
	\end{frame}
	
	\section{Modele GARCH}
	
	\begin{frame}{Stylizowane fakty}
		\textbf{Stylizowane fakty dla stóp zwrotu:}
		\begin{itemize}
			\item Rozkłady stóp zrotu mają grubsze ogony niż rozkład normalny.
			\item Wartości stóp zwrotu są nieskorelowane, ale ich kwadraty są skorelowane.
			\item Duże zmianny wartości stóp zwrotu następują często po wcześniejszych dużych zmianach.
		\end{itemize}
	\end{frame}
	
	\begin{frame}{Model ARCH}
		\begin{block}{\textbf{Model ARCH}}
			Powiemy, że szereg czasowy $r_t$ jest procesem $\mbox{ARCH}\left(p\right),$ jeżeli \begin{eqnarray*}
				r_t & = & \sigma_t \epsilon_t,\\
				\sigma^2_t & = & \alpha_0 + \sum_{j=1}^{p} \alpha_j r_{t-j}^2,
			\end{eqnarray*} gdzie
			\begin{itemize}
				\item $\epsilon_t \mbox{ iid } \mbox{WN}\left(0, 1\right),$
				\item $\alpha_0 > 0,$
				\item $\alpha_j \geq 0$ dla $j=1,2,\ldots,p.$
			\end{itemize}
		\end{block}
		\begin{alert}{\textbf{Uwaga.}}
			Model $\mbox{ARCH}$ jest często wykorzystywany do opisu reszt w modelach $\mbox{ARMA}.$
		\end{alert}
	\end{frame}
	
	\begin{frame}{Stacjonarność}
		\begin{block}{\textbf{Stacjonarność}}
			Stacjonarne szeregi czasowe typu $\mbox{ARCH}\left(p\right)$ spełniają nierówność \[
				\sum_{j=1}^{p} \alpha_j < 1.
			\]
		\end{block}
	\end{frame}
	
	\begin{frame}{Warunkowa wariancja}
		Niech $\boldsymbol{\mathcal{R}}_{\tau}$ będzie zbiorem wszystkich informacji o szeregu
		czasowym $r_t$ do chwili czasu $\tau.$ Wówczas: \begin{eqnarray*}
			\mbox{Var}\left( r_t \mid \boldsymbol{\mathcal{R}}_{t-1} \right) & = &
				\mathbb{E}\left( r_t^2 \mid \boldsymbol{\mathcal{R}}_{t-1} \right) -
				\mathbb{E}^2\left( r_t \mid \boldsymbol{\mathcal{R}}_{t-1} \right),\\
			\mbox{Var}\left( r_t \mid \boldsymbol{\mathcal{R}}_{t-1} \right) & = &
				\mathbb{E}\left( \sigma_t^2 \epsilon_t^2 \mid \boldsymbol{\mathcal{R}}_{t-1} \right) -
				\mathbb{E}^2\left( \sigma_t \epsilon_t \mid \boldsymbol{\mathcal{R}}_{t-1} \right),\\
			\mbox{Var}\left( r_t \mid \boldsymbol{\mathcal{R}}_{t-1} \right) & = &
				\sigma_t^2 \mathbb{E}\left( \epsilon_t^2 \mid \boldsymbol{\mathcal{R}}_{t-1} \right) -
				\sigma_t^2 \mathbb{E}^2\left( \epsilon_t \mid \boldsymbol{\mathcal{R}}_{t-1} \right),\\
			\mbox{Var}\left( r_t \mid \boldsymbol{\mathcal{R}}_{t-1} \right) & = &
				\sigma_t^2 \mathbb{E}\left( \epsilon_t^2 \right) -
				\sigma_t^2 \mathbb{E}^2\left( \epsilon_t \right),\\
			\mbox{Var}\left( r_t \mid \boldsymbol{\mathcal{R}}_{t-1} \right) & = &
				\sigma_t^2.
		\end{eqnarray*}
		\begin{alert}{\textbf{Wniosek.}}
			$\sigma_t^2$ jest warunkową wariancją procesu $r_t$ pod warunkiem przeszłości
			$\boldsymbol{\mathcal{R}}_{t-1}.$
		\end{alert}
	\end{frame}
	
	\begin{frame}{Efekt ARCH}
		\framesubtitle{Test Engle'a}
		Szacujemy parametry modelu: \[
			r_t^2 = \alpha_0 + \sum_{j=1}^p \alpha_j r_{t-j}^2 + \eta_t, \eta_t \mbox{ iid } \mbox{WN}\left(0, s\right).
		\]
		\\~\\
		Testujemy hipotezę postaci: \[
			H_0: \sum_{j=1}^p \alpha_j^2 = 0.
		\]
		\\~\\
		Statystyka testowa przyjmuje postać: \[
			LM = T R^2 \rightarrow \chi^2_p,
		\] gdzie $R^2$ to współczynnik determinacji modelu wyjściowego.
	\end{frame}
	
	\begin{frame}{Efekt ARCH}
		\framesubtitle{Test McLeoda-Li}
		Badamy szereg czasowy kwadratów wartości szeregu czsowego wyjściowego $r_t.$
		\\~\\
		Testujemy hipotezę postaci \[
			H_{0}: \sum_{j=1}^{h} \rho^{2}_j = 0.
		\]
		\\~\\
		Statystyka testowa przyjmuje postać: \[
			T_{LB} = n \left(n + 2\right) \sum_{j=1}^{h}
				\frac{\hat{\rho}^{2}_j}{n - j} \rightarrow \chi^{2}_{h},
		\] gdzie \[
			\hat{\rho_j} = \frac{\sum_{t=h+1}^T r_t^2 r_{t-h}^2}{\sum_{t=1}^T r_t^4}.
		\]
	\end{frame}
	
	\begin{frame}{Model GARCH}
		\begin{block}{\textbf{Model GARCH}}
			Powiemy, że szereg czasowy $r_t$ jest procesem $\mbox{GARCH}\left(p,q\right),$ jeżeli \begin{eqnarray*}
				r_t & = & \sigma_t \epsilon_t,\\
				\sigma^2_t & = & \alpha_0 + \sum_{j=1}^{q} \alpha_j r_{t-j}^2 + \sum_{j=1}^{p} \beta_j \sigma_{t-j}^2,
			\end{eqnarray*} gdzie
			\begin{itemize}
				\item $\epsilon_t \mbox{ iid } \mbox{WN}\left(0, 1\right),$
				\item $\alpha_0 > 0,$
				\item $\alpha_j \geq 0$ dla $j=1,2,\ldots,q,$
				\item $\beta_j \geq 0$ dla $j=1,2,\ldots,p.$
			\end{itemize}
		\end{block}
		\begin{alert}{\textbf{Uwaga.}}
			Model $\mbox{GARCH}$ jest często wykorzystywany do opisu reszt w modelach $\mbox{ARMA}.$
		\end{alert}
	\end{frame}
	
	\begin{frame}{Właściwości}
		\begin{itemize}
			\item Przy spełnieniu odpowiednich założeń model $\mbox{GARCH}$ można sprowadzić
				do postaci modelu $\mbox{ARCH}.$
			\item Model $\mbox{GARCH}\left(p,q\right)$ można przedstawić w postaci modelu
				$\mbox{ARMA}\left(m,p\right)$ dla szeregu $r_t^2,$ gdzie $m=\max\left(p,q\right).$
		\end{itemize}
		\begin{block}{\textbf{Stacjonarność}}
			Stacjonarne szeregi czasowe typu $\mbox{GARCH}\left(p,q\right)$ spełniają nierówność \[
				\sum_{j=1}^{q} \alpha_j + \sum_{j=1}^{p} \beta_j < 1.
			\]
		\end{block}
	\end{frame}
	
	\begin{frame}{Estymacja}
		Estymacja modeli klasy GARCH odbywa się z wykorzystaniem metody
		największej wiarogodności. Maksymalizuje się funkcję log-wiarogodności \[
			L_T\left(\boldsymbol{\theta}\right) =
				\sum_{t=1}^{T} \ell_t\left(\boldsymbol{\theta}\right),
		\] gdzie
		\begin{itemize}
			\item $\boldsymbol{\theta}$ --- wektor wszystkich parametrów modelu,
			\item $\ell_t\left(\boldsymbol{\theta}\right)$ --- funkcja log-wiarogodności
				dla obserwacji z chwili czasu $t.$
		\end{itemize}
	\end{frame}
	
	\begin{frame}{Estymacja}
		\framesubtitle{Przypadek reszt o rozkładzie normalnym}
		
		Niech dane będzie równanie \[
			r_t = \sigma_t \epsilon_t,
		\] gdzie $\epsilon_t \mbox{ iid } \mathcal{N}\left(0, 1\right).$
		\\~\\
		Funkcja log-wiarogodności dla tego równania w chwili czasu $t$ wyraża się
		wzorem \[
			\ell_t\left(\boldsymbol{\theta}\right) = 
				\log \frac{1}{\sqrt{2\pi}\sigma_t} e^{-\frac{r_t^2}{2\sigma_t^2}} = 
				-\frac{r_t^2}{2\sigma_t^2} - \log \sigma_t - \log \sqrt{2\pi}.
		\]
		
		Finalna postać tej funkcji zależy od sposobu zdefiniowania
		warunkowej wariancji $\sigma_t^2$ w danym modelu klasy GARCH.
	\end{frame}
	
	\begin{frame}{Ocena jakości modelu}
		\framesubtitle{Idea}
		\begin{block}{\textbf{Idea}}
			Inwestor jest bardziej zainteresowany poznaniem kierunku zmian kursu
			instrumentu finansowego, niż dokładną wartością tego kursu.
		\end{block}
	\end{frame}
	
	\begin{frame}{Ocena jakości modelu}
		\framesubtitle{Miary zgodności kierunku zmian stopy zwrotu}
		Niech:
		\begin{itemize}
			\item $r_t$ --- rzeczywista wartość stopy zwrotu w chwili czasu $t,$
			\item $\hat{r}_t$ --- predykowana wartość stopy zwrotu w chwili czasu $t.$
		\end{itemize}
		Podstawowe miary zgodności kierunku zmian stopy zwrotu to:
		\begin{eqnarray*}
			Q_1 & = & \frac{
				\#\left\{t: r_t \hat{r}_t > 0\right\}
			}{
				\#\left\{t: r_t \hat{r}_t \neq 0\right\}
			},\\
			Q_2 & = & \frac{
				\#\left\{t: r_{t-1} r_t < 0 \wedge r_t \hat{r}_t > 0\right\}
			}{
				\#\left\{t: r_{t-1} r_t < 0 \wedge r_t \hat{r}_t \neq 0\right\}
			}.
		\end{eqnarray*}
	\end{frame}
	
	\begin{frame}{Ocena jakości modelu}
		\framesubtitle{Modyfikacje miary $Q_1$}
		
		Uwzględniając koszta transakcji w wysokości $g$ możemy miarę $Q_1$
		zmodyfikować w następujący sposób:\[
			Q_1^{\prime} = \frac{
				\#\left\{t: r_t > g \wedge r_t \hat{r}_t > 0\right\}
			}{
				\#\left\{t: r_t > g \wedge r_t \hat{r}_t \neq 0\right\}
			}.
		\]
		
		W przypadku modeli pozwalających szacować wyłącznie zmienność stopy zwortu
		$\sigma_t$ (np. modele GARCH bez części ARIMA) możemy zastosować następującą
		modyfikację miary $Q_1:$ \[
			Q_1^{x} = \frac{
				\#\left\{t: r_t \Delta \hat{\sigma}_t < 0\right\}
			}{
				\#\left\{t: r_t \Delta \hat{\sigma}_t \neq 0\right\}
			}.
		\]
		
		
	\end{frame}
	
	\begin{frame}{Model GARCH-M}
	\end{frame}
	
	\begin{frame}{Model EGARCH}
	\end{frame}
	
	\begin{frame}{Model TGARCH}
	\end{frame}
	
	\section{Model ECM}
	
	\begin{frame}{Kointegracja}
		\begin{alert}{\textbf{Idea.}}
			Chcemy wiedzieć, czy bazując na pewnej grupie niestacjonarnych 
			szeregów czasowych możemy bezpiecznie zbudować model.
		\end{alert}
		\begin{block}{\textbf{Kointegracja}}
			Powiemy, że szeregi czasowe $X_{1,t}, X_{2,t}, \ldots, X_{n,t}$ są skointegrowane w stopniu $\left(d,b\right),$ jeżeli dla każdego $i=1,2,\ldots,n$ zachodzi \[
				X_{i,t} \sim I\left(d\right)
			\] oraz istnieją takie wartości $\beta_1,\beta_2,\ldots,\beta_n,$ że \[
				\beta_1 X_{1,t} + \beta_2 X_{2,t} + \ldots + \beta_n X_{n,t} \sim I\left(d - b\right).
			\]
		\end{block}
		\begin{alert}{\textbf{Intuicja.}}
			Relacje pomiędzy skointegrowanymi niestacjonarnymi szeregami 
			czasowymi pozostają w długiej perspektywie czasowej niezmienne.
		\end{alert}
	\end{frame}
	
	\begin{frame}{Testowanie kointegracji}
		\textbf{Procedura Engle'a-Grangera.}
		\begin{enumerate}
			\item Należy zweryfikować czy wszystkie analizowane szeregi czasowe 
				charakteryzuje ten sam stopień integracji.
			\item Należy zbudować model regresji liniowej wielorakiej w którym 
				jeden z analizowanych szeregów pełni rolę zmiennej objaśnianej, 
				pozosałe natomiast zmiennych objaśniających.
			\item Należy przetestować stopień integracji reszt wyznaczonego w 
				poprzednim kroku modelu.
		\end{enumerate}
	\end{frame}
	
	\begin{frame}{Model ECM}
		\begin{block}{\textbf{Model korekty błędem (ECM)}}
			\[
				\Delta y_{t} =
					\mu +
					\alpha \left(y_{t-1} - \beta_{0} - \beta_{1} x_{t-1}\right) + 
					\sum_{i=1}^{k-1} \theta_{i} \Delta y_{t-i} + 
					\sum_{i=0}^{k-1} \gamma_{i} \Delta x_{t-i} + 
					\epsilon_{t}
			\]
		\end{block}
		\textbf{Interpretacja:}
		\begin{itemize}
			\item $y_{t-1} = \beta_{0} + \beta_{1} x_{t-1}$ --- równanie równowagi
				długookresowej,
			\item $y_{t-1} - \beta_{0} - \beta_{1} x_{t-1}$ --- odchylenie od 
				równowagi długookr.,
			\item $\alpha$ --- współczynnik opisujący szybkość dostosowywania się 
				zmiennej objaśnianej do poziomu równowagi długookresowej (w 
				stabilnym modelu $\alpha < 0$).
			\item $\theta_{i}, \gamma_{i}$ --- współczynniki opisujące dynamikę
				krótkookresową.
		\end{itemize}
	\end{frame}
	
	\begin{frame}{Stosowalność}
		\begin{alert}{\textbf{Uwaga.}}
			Twierdzenie Grangera o reprezentacji gwarantuje nam możliwość
			zastosowania mechanizmu korekty błędem względem skointegrowanych
			szeregów czasowych.
		\end{alert}
	\end{frame}
	
	\begin{frame}{Estymacja}
		\begin{enumerate}
			\item Estymacja parametrów równania równowagi długookresowej \[
				y_{t-1} = \beta_{0} + \beta_{1} x_{t-1}.
			\]
			\item Skonstruowanie szeregów czasowych \begin{eqnarray*}
				\epsilon_{t} & = & y_{t} - \beta_{0} - \beta_{1} x_{t},\\
				\Delta x_{t} & = & x_{t} - x_{t-1},\\
				\Delta y_{t} & = & y_{t} - y_{t-1}.\\
			\end{eqnarray*}
			\item Estymacja parametrów równania modelu korekty błędem \[
				\Delta y_{t} =
					\mu +
					\alpha \epsilon_{t-1} + 
					\sum_{i=1}^{k-1} \theta_{i} \Delta y_{t-i} + 
					\sum_{i=0}^{k-1} \gamma_{i} \Delta x_{t-i} + 
					\epsilon_{t}
			\]
		\end{enumerate}
	\end{frame}

	\section*{}

	\begin{frame}
		\center
		\Huge \bfseries
		Pytania?
	\end{frame}

	\begin{frame}
		\center
		\Huge \bfseries
		Dziękuję za uwagę!
	\end{frame}

\end{document}
\documentclass[a4paper, 11pt]{beamer}

\usepackage{polski}
\usepackage[utf8]{inputenc}
\usepackage{units}

\mode<presentation> {
	\usetheme{Frankfurt}
	\setbeamercovered{transparent}
	\usecolortheme{default}
}

\title{Ekonometria Finansowa}
\subtitle{Teoria efektywności informacyjnej rynku}
\author{mgr Paweł Jamer\thanks{pawel.jamer@gmail.com}}
\institute[KEiS SGGW]{
	Doktorant, Katedra Ekonometrii i Statystyki SGGW\newline
	Ekspert ds. Modelowania Danych, Polskie Technologie\newline
	Konsultant Zewnętrzny, Polkomtel
}

\begin{document}

	\begin{frame}
		\titlepage
	\end{frame}
	
	\section{Błądzenie losowe}
	
	\begin{frame}{Biały szum}
		\begin{block}{\textbf{Biały szum}}
			Białym szumem nazwiemy szereg czasowy $\epsilon_t$ niezależnych zmiennych losowych o tym samym rozkładzie taki, że \begin{eqnarray*}
				\mathbb{E}\left(\epsilon_t\right) & = & 0,\\
				\mbox{Var}\left(\epsilon_t\right) & = & \sigma^2.
			\end{eqnarray*} Biały szum oznaczać będziemy symbolem $\mbox{WN}\left(0, \sigma^2\right)$.
		\end{block}
		\begin{alert}{\textbf{Uwaga}}
			Bardziej złożone modele szeregów czasowych wykorzystują biały szum do opisu niepewności pomiaru opisywanych przez nie wielkości.
		\end{alert}
	\end{frame}
	
	\begin{frame}{Błądzenie losowe}
		\begin{block}{\textbf{Błądzenie losowe (bez dryfu)}}
			Szereg czasowy $p_t$ nazwiemy błądzeniem losowym bez dryfu, jeżeli spełnia on równanie \[
				p_t = p_{t-1} + \epsilon_t,
			\] gdzie
			\begin{itemize}
				\item $\epsilon_t$ --- biały szum.
			\end{itemize}
		\end{block}
		\begin{alert}{\textbf{Uwaga.}}
			Uzupełniając powyższy wzór o niezerową stałą $\alpha$ \[
				p_t = \alpha + p_{t-1} + \epsilon_t
			\] uzyskujemy proces błądzenia losowego z dryfem.
		\end{alert}
	\end{frame}
	
	\begin{frame}{Błądzenie losowe}
		\framesubtitle{...jako generator cen instrumentu finansowego}
		\begin{block}{\textbf{Hipoteza}}
			Cena instrumentu finansowego $p_t$ jest błądzeniem losowym.
		\end{block}
		Rozważmy model \[
			p_t = \alpha + \rho p_{t-1} + \epsilon_t.
		\]
		Prawdziwość powyższej hipotezy jest równoznaczna z tym, że:
		\begin{itemize}
			\item $\rho$ statystycznie nie różni się od jedności,
			\item $\epsilon_t$ jest białym szumem.
		\end{itemize}
		Ponadto, jeżeli zachodzi:
		\begin{itemize}
			\item $\alpha = 0,$ to $p_t$ jest błądzeniem losowym bez dryfu,
			\item $\alpha \neq 0,$ to $p_t$ jest błądzeniem losowym z dryfem.
		\end{itemize}
	\end{frame}
	
	\begin{frame}{Właściwości błądzenia losowego}
		\begin{columns}[onlytextwidth]
			\begin{column}{0.5\textwidth}
				\textbf{Błądzenie losowe bez dryfu} \[
					p_t = p_{t-1} + \epsilon_t,
				\] \[
					p_t = p_0 + \sum_{h=0}^{t} \epsilon_{t-h},
				\] \[
					\mathbb{E}\left(p_t\right) = p_0,
				\] \[
					\mbox{Var}\left(p_t\right) = t \sigma^2_{\epsilon_t}.
				\]
			\end{column}
			\begin{column}{0.5\textwidth}
				\textbf{Błądzenie losowe z dryfem} \[
					p_t = \alpha + p_{t-1} + \epsilon_t,
				\] \[
					p_t = p_0 + t \alpha + \sum_{h=0}^{t} \epsilon_{t-h},
				\] \[
					\mathbb{E}\left(p_t\right) = p_0 + t \alpha,
				\] \[
					\mbox{Var}\left(p_t\right) = t \sigma^2_{\epsilon_t}.
				\]
			\end{column}
		\end{columns}
	\end{frame}
	
	\begin{frame}{Stopy zwrotu instrumentów finansowych}
		Rozważmy model błądzenia losowego bez dryftu dla logarytmu cen pewnego instrumentu finansowego \[
			\log\left(p_t\right) = \log\left(p_{t-1}\right) + \epsilon_t.
		\] Model ten przekształcić możemy do postaci \[
			r_t = \log\left(\frac{p_t}{p_{t-1}}\right) = \epsilon_t.
		\]
		\begin{alert}{\textbf{Uwaga.}}
			Badanie czy logarytm cen $p_t$ instrumentu finansowego jest błądzeniem losowym
			sprowadza się do ustalenia, czy logarytmiczne stopy zwrotu $r_t$ tego instrumentu są białym szumem.
		\end{alert}
	\end{frame}
	
	\begin{frame}{Krytyka}
		Optymalna prognoza ceny instrumentu finansowego na okres przyszły, to przyjęcie ceny tego instrumentu z okresu bieżącego.
		\\~\\
		Nie uwzględnia się rentowności zależnej od ryzyka.
	\end{frame}
	
	\section{Racjonalność}
	
	\begin{frame}{Racjonalność zachowań}
		\begin{itemize}
			\item Uczestnicy rynku działają racjonalnie...
			\begin{itemize}
				\item ...znając cały zbiór informacji...
				\item ...dysponując takim samym zbiorem narzędzi analiy rynku.x
			\end{itemize}
			\item Uczestnicy rynku dążą do...
			\begin{itemize}
				\item ...maksymalizacji zysku przy ustalonym z góry poziomie ryzyka.
				\item ...minimalizacji ryzyka przy ustalonym z góry poziomie zysku.
			\end{itemize}
		\end{itemize}
	\end{frame}
	
	\begin{frame}{Formy racjonalności zachowań}
		\begin{itemize}
			\item \textbf{Racjonalność instrumentalna} --- uczestnik rynku posiadający
				nieograniczone zdolności i nieograniczony dostęp do informacji dąży
				do optymalizaji swojej funkcji celu.
			\item \textbf{Racjonalność kognitywna} --- uczestnik rynku posiadający zdoność
				konfrontowania posiadanych informacji z otoczeniem dąży do optymalizacji swojej
				funkcji celu.
			\item \textbf{Racjonalność ograniczona} --- uczestnik rynku posiadający
				ograniczone zdolności i ograniczony dostęp do informacji dąży do osiągnięcia
				zadowalającego dla siebie poziomu swojej funkcji celu.
			\end{itemize}
	\end{frame}
	
	\begin{frame}{Racjonalność przewidywań}
		\begin{block}{\textbf{Racjonalność przewidywań}}
			Uczestnik rynku antycypuje przyszłość wykorzystując cały zbiór dostępnych mu informacji w najlepszy
			ze znanych mu i możliwych do zastosowania sposobów.
		\end{block}
		\textbf{Warunki konieczne zachodzenia racjonalności przewidywań:}
		\begin{itemize}
			\item uczestnik rynku optymalnie specyfikuje model zależności zmiennej objaśnianej
				od zbioru zmiennych objaśniających,
			\item uczestnik rynku posiada wystarczająco duży zbiór przeszłych wartości wszystkich
				występujących w modelu zmiennych,
			\item uczestnik rynku wykorzystuje metody estymacji prowadzące do uzyskania
				estymatorów nieobciążonych.
		\end{itemize}
	\end{frame}
	
	\begin{frame}{Racjonalność przewidywań}
		\begin{block}{\textbf{Racjonalne przewidywanie}}
			Racjonalna predykcja wartości jaką $r_t$ przyjmie w chwili $t+1$ dokonywana w chwili $t$ przy założeniu
			posiadania wszystkich niezbędnych do dokonania predykcji informacji $I_t,$ to \[
				\mathbb{E}\left(r_{t+1} \mid I_t\right).
			\]
		\end{block}
		\begin{alert}{\textbf{Uwaga.}}
			W efekcie niedoskonałości rynków finansowych wartość $r_t$ w chwili $t+1$ nie będzie z reguły zgodna
			z prognozą. Nie powinna ona być jednak obciążona błędem systematycznym, tzn.: \[
				r_{t+1} = \mathbb{E}\left(r_{t+1} \mid I_t\right) + \epsilon_{t+1},
			\] gdzie $\epsilon_{t+1}$ to biały szum.
		\end{alert}
	\end{frame}
	
	\section{Rynek efektywny}
	
	\begin{frame}{Formy efektywności rynków finansowych}
		\textbf{Formy efektywności rynków finansowych.}
		\begin{itemize}
			\item \textbf{Efektywność alokacji} --- przepływ kapitału pozwalający realizować najlepsze
				z punktu widzenia danej branży oraz całej gospodarki przedsięwzięcia.
			\item \textbf{Efektywność operacyjna} --- kojarzenie przez pośredników rynku finansowego
				osób posiadających kapitał oraz potrzebujących kapitału przy zachowaniu możliwie
				optymalnych warunków.
			\item \textbf{Efektywność informacyjna} --- odzwierciedlenie w cenie instrumentu finansowego
				wszystkich związanych z nim informacji przeszłych i obecnych, jak również rozsądnych
				przewidywań dotyczących przyszłości.
		\end{itemize}
		\begin{alert}{\textbf{Uwaga.}}
			Wymienione wyżej formy efektywności są ze sobą silnie powiązane i uzupełniają się wzajemnie.
		\end{alert}
	\end{frame}
	
	\begin{frame}{Hipoteza rynku efektywnego}
		\framesubtitle{Wprowadzenie}
		\begin{block}{\textbf{Hipoteza rynku efektywnego (EMH)}}
			Łączne zachodzenie efektywności alokacji, efektywności operacyjnej oraz efektywności informacyjnej.
		\end{block}
		\textbf{Warunki wystarczające efektywności rynku:}
		\begin{itemize}
			\item racjonalność zachowań uczestników rynku,
			\item powszechny dostęp do natychmiastowej, pewnej i bezpłatnej informacji,
			\item brak opłat oraz podatków na giełdzie.
		\end{itemize}
	\end{frame}
	
	\begin{frame}{Hipoteza rynku efektywnego}
		\framesubtitle{Wnioski}
		\begin{block}{\textbf{Wniosek 1} (nieprognozowalność)}
			Zgodnie z założeniami hipotezy rynku efektywnego nie jest możliwa prognoza ceny instrumentu
			finansowego na okres przyszły.
		\end{block}
		\begin{block}{\textbf{Wniosek 2} (losowość)}
			Zmiana ceny instrumentu finansowego w okresie następnym może nastąpić tylko na skutek
			napływu nowej informacji lub zaistnienia nieprognozowalnego zdarzenia.
		\end{block}
		\begin{block}{\textbf{Wniosek 3} (stabilność)}
			Jeżeli zbiór informacji na temat instrumentu finansowego w dwóch następujących po sobie
			okresach czasu $I_t$ oraz $I_{t+1}$ nie uległ zmianie, to również cena tego instrumentu
			finansowego na rynku efektymwnym nie ulegnie zmianie, tzn. $P_t = P_{t+1}.$
		\end{block}
	\end{frame}

	\begin{frame}{Hipoteza rynku efektywnego}
		\framesubtitle{Realia rynku}
		\textbf{Inwestorzy:}
		\begin{itemize}
			\item nie dysponują takim samym zbiorem informacji,
			\item mają różne preferencje i cele,
			\item cechują się różnym poziomem wiedzy i doświadczenia,
			\item dysponują kapitałem różnej wielkości,
			\item stosują różne strategie inwestycyjne,
			\item ...
		\end{itemize}
		\begin{alert}{\textbf{Wniosek.}}
			Ceny instrumentów finansowych podlegają pewnym wahaniom spowodowanym
			podejmowaniem przez inwestorów zróżnicowanych decyzji inwestycyjnych.
		\end{alert}
	\end{frame}
	
	\begin{frame}{Hipoteza rynku efektywnego informacyjnie}
		\framesubtitle{Twierdzenie}
		\begin{block}{\textbf{Twierdzenie o efektywności rynku} (Fama)}
			Efektywność informacyjna rynku finansowego przejawia się w trzech formach.
			\begin{itemize}
				\item \textbf{Forma słaba} --- w cenie instrumentu finansowego znajdują
					odzwierciedlenie wszystkich informacje historyczne z instrumentem powiązane.
				\item \textbf{Forma półsilna} --- w cenie instrumentu finansowego znajdują
					odzwierciedlenie informacje uwzględnione w słabej formie efektywności oraz
					ogólnie dostępne informacje bieżące.
				\item \textbf{Forma silna} --- w cenie instrumentu finansowego znajdują
					odzwierciedlenie informacje uwzględnione w półsilnej formie efektywności oraz
					bieżące informacje poufne.
			\end{itemize}
		\end{block}
	\end{frame}
	
	\begin{frame}{Testowanie efektywności rynku}
		\textbf{Hipotezę o słabej efektywności rynku weryfikować można:}
		\begin{itemize}
			\item wykorzystując narzędzia analizy technicznej
			\begin{itemize}
				\item \textit{rynek dyskontuje wszystko},
			\end{itemize}
			\item stosując testy losowości
			\begin{itemize}
				\item testy autokorelacji,
				\item test ilorazów wariancji,
				\item test serii,
				\item ...
			\end{itemize}
		\end{itemize}
	\end{frame}
	
	\begin{frame}{Testowanie autokorelacji}
		\begin{block}{\textbf{Cel}}
			Badanie, czy stopy zwrotu instrumentu finansowego z różnych okresów
			są ze sobą powiązane.
		\end{block}
		\begin{columns}
			\begin{column}{0.5\textwidth}
				\begin{center}
					\textbf{Test Pearsona}
				\end{center}
				Testujemy hipotezę \[
					\begin{cases}
						H_{0}: & \rho_{i}=0,\\
						H_{1}: & \rho_{i}\neq0
					\end{cases}
				\]
				wykorzystując w tym celu test \[
					t=\frac{\hat{\rho}_i}{\sqrt{\frac{1-\hat{\rho}^2_i}{T-i-2}}} \sim t^{\left[T-i-2\right]}.
				\]
			\end{column}
			\begin{column}{0.5\textwidth}
				\begin{center}
					\textbf{Test Ljunga-Boxa}
				\end{center}
				Testujemy hipotezę \[
					\begin{cases}
						H_{0}: & \sum_{i=1}^{h} \rho_{i}^{2}=0,\\
						H_{1}: & \sum_{i=1}^{h} \rho_{i}^{2}>0
					\end{cases}
				\]
				wykorzystując w tym celu test \[
					Q = T \left(T + 2\right) \sum_{i=1}^h \frac{\hat{\rho}^2_i}{T-i} \sim \chi^2_{h}.
				\]
			\end{column}
		\end{columns}
	\end{frame}
	
	\begin{frame}{Test ilorazów wariancji}
		\begin{block}{\textbf{Cel}}
			Badanie, czy stopy zwrotu instrumentu finansowego zmieniają się
			w losowy sposób.
		\end{block}
		Testujemy hipotezę \[
				H_{0}: \mbox{błądzenie losowe},
		\] wykorzystując w tym celu test \[
			SVR_h = \sqrt{\frac{3hT}{2\left(2h-1\right)\left(h-1\right)}} \left(VR_h - 1\right) \sim \mathcal{N}\left(0,1\right),
		\] gdzie
		\begin{itemize}
			\item $VR_h = \frac{S^2\left(r_t+r_{t-1}+\ldots+r_{t-k+1}\right)}{hS^2\left(r_t\right)},$
			\item $S^2\left(r_t\right)$ --- wariancja z szeregu czasowego $r_t.$
		\end{itemize}
	\end{frame}
	
	\begin{frame}{Test serii}
		\begin{block}{\textbf{Cel}}
			Badanie, czy stopy zwrotu instrumentu finansowego mają charakter
			losowy.
		\end{block}
		Testujemy hipotezę $H_{0}: \mbox{dane mają charakter losowy},$ 
		wykorzystując w tym celu test \[
			U=\frac{K-\mathbb{E}\left(K\right)}{\sqrt{\mbox{Var}\left(K\right)}} \sim \mathcal{N}\left(0,1\right),
		\] gdzie
		\begin{itemize}
			\item $K$ --- liczba wszystkich serii obserwacji nieujemnych oraz obserwacji ujemnych,
			\item $\mathbb{E}\left(K\right) = \frac{2 n_1 n_2 + n}{n}$ oraz
				$\mbox{Var}\left(K\right) = \frac{2 n_1 n_2 \left(2 n_1 n_2 - n\right)}{\left(n - 1\right) n^2},$
			\begin{itemize}
				\item $n, n_1, n_2$ --- liczba odpowiednio wszystkich obserwacji, nieujemnych obserwacji, ujemnych obserwacji w szeregu.
			\end{itemize}
		\end{itemize}
	\end{frame}
	
	\section{Efekty kalendarzowe}
	
	\begin{frame}{Wprowadzenie}
		\begin{block}{\textbf{Idea}}
			Na wschodzących rynkach finansowych zaobserwować można różne 
			zachowanie instrumentów finansowych w różnych dniach tygodnia, okresach
			miesiąca lub fragmentach roku.
		\end{block}
		\begin{block}{\textbf{Problem}}
			Występowanie efeków kalendarzowych na rynku finansowym stwarza warunki 
			do budowania strategii inwestycyjnych pozwalających w krótkim czasie 
			osiągnąć zysk bez ponoszenia ryzyka. Efekty kalendarzowe mogą zatem
			prowadzić do nieefektywności rynku finansowego.
		\end{block}
	\end{frame}
	
	\begin{frame}{Klasyfikacja}
		\begin{block}{\textbf{Efekt końca tygodnia}}
			Rentowność instrumentów finansowych jest mniejsza na początku tygodnia, a
			większa pod koniec tygodnia.
		\end{block}
		\begin{block}{\textbf{Efekt końca miesiąca}}
			Rentowność instrumentów finansowych jest większa w pierwszej połowie 
			miesiąca, a mniejsza w drugiej połowie miesiąca.
		\end{block}
		\begin{block}{\textbf{Efekt końca roku}}
			Rentowność instrumentów finansowych w grudniu maleje w stosunku do 
			średniej rentowności rocznej, natomiast rentowność instrumentów 
			finasnowych w styczniu rośnie w stosunku do średniej rentowności rocznej.
		\end{block}
	\end{frame}
	
	\begin{frame}{Test równości średnich}
		Testujemy hipotezę \[
				\begin{cases}
					H_{0}: & \mathbb{E}\left(y_i\right) = \mathbb{E}\left(y_j\right)\\
					H_{1}: & \mathbb{E}\left(y_i\right) \neq \mathbb{E}\left(y_j\right),
				\end{cases}
			\]
		wykorzystując w tym celu statystykę testową: \[
				u=\frac{\overline{y}_i - \overline{y}_j}{ \sqrt{\frac{S_i^2}{T_i} + \frac{S_j^2}{T_j}}} ,
			\]
		gdzie:
		\begin{itemize}
			\item $\overline{y}_i$ --- średnia z obserwacji w szeregu z $i$-tego dnia tygodnia,
			\item $S_i^2$ --- wariancja obserwacji w szeregu z $i$-tego dnia tygodnia,
			\item $T_i$ --- liczba obserwacji w szeregu z $i$-tego dnia tygodnia,
			\item $i,j$ --- poniedziałek, wtorek, środa, czwartek, piątek, przy czym $i \neq j$.
		\end{itemize}
		Dla dużej próby $u \sim \mathcal{N}\left(0,1\right).$
	\end{frame}
	
	\begin{frame}{Test równości wariancji}
		Testujemy hipotezę \[
				\begin{cases}
					H_{0}: & \sigma_1^2 = \sigma_2^2\\
					H_{1}: & \sigma_1^2 > \sigma_2^2,
				\end{cases}
			\]
		wykorzystując w tym celu statystykę testową: \[
				F=\frac{\max\left(S_i^2, S_j^2\right)}{\min\left(S_i^2, S_j^2\right)}  \sim \mathbb{F}^{[T_i-1, T_j-1]},
			\]
		gdzie:
		\begin{itemize}
			\item $S_1^2$, $S_2^2$ --- wariancje wyznaczone z pierwszej i drugiej próby,
			\item $T_1$, $T_2$ --- liczba obserwacji z pierwszej i drugiej grupy,
			\item $i,j$ --- poniedziałek, wtorek, środa, czwartek, piątek, przy czym $i \neq j$.
		\end{itemize}
	\end{frame}	
	
	\begin{frame}{Test istotności współczynników regresji liniowej}
		Rozważmy następujący model: \[
			y_t = \alpha_1 PON_t + \alpha_2 WT_t + \alpha_3 SR_t + \alpha_4 CZW_t + \alpha_5 PT_t + \epsilon_t
			\]
		gdzie:
		\begin{itemize}
			\item $y_t$ --- obserwacja szeregu w okresie $t$,
			\item $D_t =
				\left\{ \begin{array}{ll} 
					1,& \mbox{gdy rozważamy oberwacje z dnia } D\\
					0,& \mbox{wpp.}
				\end{array} \right.$, przy czym  $D \in \left(PON, WT, SR, CZW, PT\right)$,
			\item $\alpha_i$ --- współczynniki regresji dla $i \in 1,2,\ldots,5$,
			\item $\epsilon \sim \mathcal{N}(0,\sigma^2)$
		\end{itemize}
	\end{frame}
	
	\begin{frame}{Test istotności współczynników regresji liniowej}
		\begin{block}{\textbf{Cel}}
			Badanie, czy zmienna objaśniająca ma istotny wpływ na zmienną objaśnianą $y$.
		\end{block}
		Testujemy hipotezę \[  
		                    H_{0}: \alpha_i = 0
		                   \]
		wobec hipotez alternatywnych \[ 
		                              H_{1}: \alpha_i \neq 0, H_{1}: \alpha_i < 0, H_{1}: \alpha_i > 0,
		                             \]
		wykorzystując w tym celu statystykę testową \[
			t=\frac{\hat{\alpha_i}}{S\left(\hat{\alpha_i}\right)} \sim t^{\left[n-k\right]},
		\] gdzie
		\begin{itemize}
			\item $\hat{\alpha_i}$ --- wyestymowana wartość parametru $\alpha_i$,
			\item $S(\hat{\alpha_i})$ --- błąd szacunku.
		\end{itemize}
	\end{frame}
	
	\section*{}
	
	\begin{frame}
		\center
		\Huge \bfseries
		Pytania?
	\end{frame}

	\begin{frame}
		\center
		\Huge \bfseries
		Dziękuję za uwagę!
	\end{frame}

\end{document}
\documentclass[a4paper, 11pt]{beamer}

\usepackage{polski}
\usepackage[utf8]{inputenc}
\usepackage{units}

\mode<presentation> {
	\usetheme{Frankfurt}
	\setbeamercovered{transparent}
	\usecolortheme{default}
}

\title{Ekonometria Finansowa}
\subtitle{Teoria efektywności informacyjnej rynku}
\author{mgr Paweł Jamer\thanks{pawel.jamer@gmail.com}}

\begin{document}

	\begin{frame}
		\titlepage
	\end{frame}
	
	\section{Biały szum}
	
	\begin{frame}{Biały szum}
		\begin{block}{\textbf{Biały szum}}
			Białym szumem nazwiemy szereg czasowy $\epsilon_t$ niezależnych zmiennych losowych o tym samym rozkładzie taki, że \begin{eqnarray*}
				\mathbb{E}\left(\epsilon_t\right) & = & 0,\\
				\mbox{Var}\left(\epsilon_t\right) & = & \sigma^2.
			\end{eqnarray*} Biały szum oznaczać będziemy symbolem $\mbox{WN}\left(0, \sigma^2\right)$.
		\end{block}
		\begin{alert}{\textbf{Uwaga}}
			Bardziej złożone modele szeregów czasowych wykorzystują biały szum do opisu niepewności pomiaru opisywanych przez nie wielkości.
		\end{alert}
	\end{frame}
	
	\section{Błądzenie losowe}
	
	\begin{frame}{Błądzenie losowe}
		\begin{block}{\textbf{Błądzenie losowe (bez dryftu)}}
			Szereg czasowy $p_t$ nazwiemy błądzeniem losowym bez dryftu, jeżeli spełnia on równanie \[
				p_t = p_{t-1} + \epsilon_t,
			\] gdzie
			\begin{itemize}
				\item $\epsilon_t$ --- biały szum.
			\end{itemize}
		\end{block}
		\begin{alert}{\textbf{Uwaga.}}
			Uzupełniając powyższy wzór o niezerową stałą $\alpha$ \[
				p_t = \alpha + p_{t-1} + \epsilon_t
			\] uzyskujemy proces błądzenia losowego z dryftem.
		\end{alert}
	\end{frame}
	
	\begin{frame}{Ceny instrumentów finansowych}
		\begin{block}{\textbf{Hipoteza}}
			Cena instrumentu finansowego $p_t$ jest błądzeniem losowym.
		\end{block}
		Rozważmy model \[
			p_t = \alpha + \rho p_{t-1} + \epsilon_t.
		\]
		Prawdziwość powyższej hipotezy jest równoznaczna z tym, że:
		\begin{itemize}
			\item $\hat{\rho}$ statystycznie nie różni się od jedności,
			\item $\epsilon_t$ jest białym szumem.
		\end{itemize}
		Ponadto, jeżeli na zadanym poziomie istotności zachodzi:
		\begin{itemize}
			\item $\hat{\alpha} = 0,$ to $p_t$ jest błądzeniem losowym bez dryfu,
			\item $\hat{\alpha} \neq 0,$ to $p_t$ jest błądzeniem losowym z dryfem.
		\end{itemize}
		\begin{alert}{\textbf{Uwaga.}}
			Z powodu możliwej niestacjonarności $p_t$ estymacja powyższego równania
			jest problematyczna.
		\end{alert}
	\end{frame}
	
	\begin{frame}{Właściwości błądzenia losowego}
		\begin{columns}[onlytextwidth]
			\begin{column}{0.5\textwidth}
				\textbf{Błądzenie losowe bez dryftu} \[
					p_t = p_{t-1} + \epsilon_t,
				\] \[
					p_t = p_0 + \sum_{h=0}^{t} \epsilon_{t-h},
				\] \[
					\mathbb{E}\left(p_t\right) = p_0,
				\] \[
					\mbox{Var}\left(p_t\right) = t \sigma^2_{\epsilon_t}.
				\]
			\end{column}
			\begin{column}{0.5\textwidth}
				\textbf{Błądzenie losowe z dryftem} \[
					p_t = \alpha + p_{t-1} + \epsilon_t,
				\] \[
					p_t = p_0 + t \alpha + \sum_{h=0}^{t} \epsilon_{t-h},
				\] \[
					\mathbb{E}\left(p_t\right) = p_0 + t \alpha,
				\] \[
					\mbox{Var}\left(p_t\right) = t \sigma^2_{\epsilon_t}.
				\]
			\end{column}
		\end{columns}
	\end{frame}
	
	\begin{frame}{Stopy zwrotu instrumentów finansowych}
		Rozważmy model błądzenia losowego bez dryftu dla logarytmu cen pewnego instrumentu finansowego \[
			\log\left(p_t\right) = \log\left(p_{t-1}\right) + \epsilon_t.
		\] Model ten przekształcić możemy do postaci \[
			r_t = \log\left(\frac{p_t}{p_{t-1}}\right) = \epsilon_t.
		\]
		\begin{alert}{\textbf{Uwaga.}}
			Badanie czy logarytm cen $p_t$ instrumentu finansowego jest błądzeniem losowym
			sprowadza się do ustalenia, czy logarytmiczne stopy zwrotu $r_t$ tego instrumentu są białym szumem.
		\end{alert}
	\end{frame}
	
	\begin{frame}{Krytyka}
		Optymalna prognoza ceny instrumentu finansowego na okres przyszły, to przyjęcie ceny tego instrumentu z okresu bieżącego.
		\\~\\
		Nie uwzględnia się rentowności zależnej od ryzyka.
	\end{frame}
	
	\section{Racjonalność}
	
	\begin{frame}{Racjonalność zachowań}
		\textbf{Racjonalność zachowań:}
		\begin{itemize}
			\item uczestnicy rynku działają racjonalnie znając cały zbiór informacji,
			\item uczestnicy rynku dysponują takim samym zbiorem narzędzi analizy rynku,
			\item uczestnicy rynku dążą do maksymalizacji zysku przy ustalonym z góry poziomie ryzyka
				lub minimalizacji ryzyka przy ustalonym z góry poziomie zysku.
		\end{itemize}
	\end{frame}
	
	\begin{frame}{Formy racjonalności zachowań}
		\textbf{Formy racjonalności zachowań:}
		\begin{itemize}
			\item \textbf{racjonalność instrumentalna} --- uczestnik rynku o nieograniczonych zdolnościach
				dąży do optymalizaji swojej funkcji celu w warunkach wolnego od opłat dostępu do pełnej i pewnej
				informacji.
			\item \textbf{racjonalność kognitywna} --- uczestnik rynku konfrontuje posiadane informacje z
				realiami swojego otoczenia.
			\item \textbf{racjonalność ograniczona} --- uczestnik rynku dysponuje ograniczonymi zdolnościami
				oraz ograniczonym dostępem do informacji. Zadowala się on osiągnięciem wyników, uznawanych przez
				siebie za dostateczne.
		\end{itemize}
	\end{frame}
	
	\begin{frame}{Racjonalność przewidywań}
		\begin{block}{\textbf{Racjonalność przewidywań}}
			Uczestnik rynku antycypuje przyszłość wykorzystując cały zbiór dostępnych mu informacji w najlepszy
			ze znanych mu i możliwych do zastosowania sposobów.
		\end{block}
		\textbf{Warunki konieczne zachodzenia racjonalności przewidywań:}
		\begin{itemize}
			\item uczestnik rynku optymalnie specyfikuje model zależności zmiennej predykowanej od zbioru zmiennych
				predykcyjnych,
			\item uczestnik rynku posiada wystarczający zbiór informacji o pszeszłych wartościach wszystkich
				występujących w modelu zmiennych,
			\item uczestnik rynku dokonuje predykcji z wykorzystaniem metod estymacji prowadzących do uzyskania
				estymatorów nieobciążonych.
		\end{itemize}
	\end{frame}
	
	\begin{frame}{Racjonalność przewidywań}
		\begin{block}{\textbf{Racjonalne przewidywanie}}
			Racjonalna predykcja wartości jaką $r_t$ przyjmie w chwili $t+1$ dokonywana w chwili $t$ przy założeniu
			posiadania wszystkich niezbędnych do dokonania predykcji informacji $I_t,$ to \[
				\mathbb{E}\left(r_{t+1} \mid I_t\right).
			\]
		\end{block}
		\begin{alert}{\textbf{Uwaga.}}
			W efekcie niedoskonałości rynków finansowych wartość $r_t$ w chwili $t+1$ nie będzie z reguły zgodna
			z prognozą. Nie powinna ona być jednak obciążona błędem systematycznym, tzn.: \[
				r_{t+1} = \mathbb{E}\left(r_{t+1} \mid I_t\right) + \epsilon_{t+1},
			\] gdzie $\epsilon_{t+1}$ to biały szum.
		\end{alert}
	\end{frame}
	
	\section{Rynek efektywny}
	
	\begin{frame}{Formy efektywności rynków finansowych}
		\textbf{Formy efektywności rynków finansowych:}
		\begin{itemize}
			\item \textbf{efektywność alokacji} --- przepływ kapitału pozwalający realizować
				przedsięwzięcia najbardziej efektywne i zapewniające stabilny oraz odpowienio szybki
				rozwój gospodarki.
			\item \textbf{efektywność operacyjna} --- kojarzenie przez pośredników rynku finansowego
				osób posiadających kapitał oraz potrzebujących kapitału, w sposób satysfakcjonujący dla
				obu stron oraz w zamian za możliwie niskie opłaty.
			\item \textbf{efektywność informacyjna} --- odzwierciedlenie w cenie instrumentu finansowego
				wszystkich związanych z nim informacji przeszłych i obecnych, jak również rozsądnych
				przewidywań dotyczących przyszłości.
		\end{itemize}
		\begin{alert}{\textbf{Uwaga.}}
			Wymienione wyżej formy efektywności są ze sobą silnie powiązane i uzupełniają się wzajemnie.
		\end{alert}
	\end{frame}
	
	\begin{frame}{Hipoteza rynku efektywnego}
		\framesubtitle{Wprowadzenie}
		\begin{block}{\textbf{Hipoteza rynku efektywnego (EMH)}}
			Łączne zachodzenie efektywności alokacji, efektywności operacyjnej oraz efektywności informacyjnej.
		\end{block}
		\textbf{Warunki wystarczające efektywności rynku:}
		\begin{itemize}
			\item racjonalność zachowań uczestników rynku,
			\item powszechny dostęp do natychmiastowej, pewnej i bezpłatnej informacji,
			\item brak opłat oraz podatków na giełdzie.
		\end{itemize}
		\begin{alert}{\textbf{Wniosek.}}
			Ceny instrumentów finansowych stanowią wyraz całości przeszłej, teraźniejszej oraz racjonalnie
			antycypowanej przyszłej informacji na ich temat. Nie jest zatem możliwa prognoza cen instrumentów
			na okres następny.
		\end{alert}
	\end{frame}
	
	\begin{frame}{Hipoteza rynku efektywnego}
		\framesubtitle{Realia rynku}
		\textbf{Inwestorzy:}
		\begin{itemize}
			\item nie dysponują takim samym zbiorem informacji,
			\item mają różne preferencje i cele,
			\item cechują się różnym poziomem wiedzy i doświadczenia,
			\item dysponują kapitałem różnej wielkości,
			\item stosują różne strategie inwestycyjne,
			\item ...
		\end{itemize}
	\end{frame}
	
	\begin{frame}{Hipoteza rynku efektywnego informacyjnie}
		\framesubtitle{Twierdzenie}
		\begin{block}{\textbf{Twierdzenie o efektywności rynku}}
			Efektywność informacyjna rynku finansowego przejawia się w trzech formach.
			\begin{itemize}
				\item \textbf{forma słaba} --- w cenie instrumentu finansowego znajdują
					odzwierciedlenie wszystkich informacje historyczne z instrumentem powiązane.
				\item \textbf{forma półsilna} --- w cenie instrumentu finansowego znajdują
					odzwierciedlenie informacje uwzględnione w słabej formie efektywności oraz
					ogólnie dostępne informacje bieżące
				\item \textbf{forma silna} --- w cenie instrumentu finansowego znajdują
					odzwierciedlenie informacje uwzględnione w półsilnej formie efektywności oraz
					bieżące informacje poufne.
			\end{itemize}
		\end{block}
	\end{frame}
	
	\begin{frame}{Testowanie efektywności rynku}
		\textbf{Hipotezę o słabej efektywności rynku weryfikować można:}
		\begin{itemize}
			\item wykorzystując narzędzia analizy technicznej
			\begin{itemize}
				\item \textit{rynek dyskontuje wszystko},
			\end{itemize}
			\item stosując testy losowości
			\begin{itemize}
				\item testy autokorelacji,
				\item test ilorazów wariancji,
				\item test serii,
				\item ...
			\end{itemize}
		\end{itemize}
	\end{frame}
	
	\begin{frame}{Testowanie autokorelacji}
		\begin{columns}
			\begin{column}{0.5\textwidth}
				\begin{center}
					\textbf{Test Pearsona}
				\end{center}
				Testujemy hipotezę \[
					\begin{cases}
						H_{0}: & \rho_{i}=0,\\
						H_{1}: & \rho_{i}\neq0
					\end{cases}
				\]
				wykorzystując w tym celu test \[
					t=\frac{\hat{\rho}_i}{\sqrt{\frac{1-\hat{\rho}^2_i}{T-i-2}}} \sim t_{T-i-2}.
				\]
			\end{column}
			\begin{column}{0.5\textwidth}
				\begin{center}
					\textbf{Test Ljunga-Boxa}
				\end{center}
				Testujemy hipotezę \[
					\begin{cases}
						H_{0}: & \sum_{i=1}^{h} \rho_{i}^{2}=0,\\
						H_{1}: & \sum_{i=1}^{h} \rho_{i}^{2}>0
					\end{cases}
				\]
				wykorzystując w tym celu test \[
					Q = T \left(T + 2\right) \sum_{i=1}^h \frac{\hat{\rho}^2_i}{T-i} \sim \chi^2_{h}.
				\]
			\end{column}
		\end{columns}
	\end{frame}
	
	\begin{frame}{Test ilorazów wariancji}
		Testujemy hipotezę \[
				H_{0}: \mbox{błądzenie przypadkowe (typu RW1)},
		\] wykorzystując w tym celu test \[
			SVR_h = \sqrt{\frac{3hT}{2\left(2h-1\right)\left(h-1\right)}} \left(VR_h - 1\right) \sim \mathcal{N}\left(0,1\right),
		\] gdzie
		\begin{itemize}
			\item $VR_h = \frac{S^2\left(r_t+r_{t-1}+\ldots+r_{t-k+1}\right)}{hS^2\left(r_t\right)}.$
		\end{itemize}
	\end{frame}
	
	\begin{frame}{Test serii}
		Testujemy hipotezę \[
				H_{0}: \mbox{dane mają charakter losowy}
		\] wykorzystując w tym celu test \[
			U=\frac{K-\mathbb{E}\left(K\right)}{\sqrt{\mbox{Var}\left(K\right)}} \sim \mathcal{N}\left(0,1\right),
		\] gdzie
		\begin{itemize}
			\item $\mathbb{E}\left(K\right) = \frac{2 n_1 n_2 + n}{n},$
			\item $\mbox{Var}\left(K\right) = \frac{2 n_1 n_2 \left(2 n_1 n_2 - n\right)}{\left(n - 1\right) n^2},$
			\item $K$ --- liczba wszystkich serii obserwacji nieujemnych oraz obserwacji ujemnych,
			\item $n, n_1, n_2$ --- liczba odpowiednio wszystkich obserwacji, nieujemnych obserwacji, ujemnych obserwacji w szeregu.
		\end{itemize}
	\end{frame}
	
	\section*{}
	
	\begin{frame}
		\center
		\Huge \bfseries
		Pytania?
	\end{frame}

	\begin{frame}
		\center
		\Huge \bfseries
		Dziękuję za uwagę!
	\end{frame}

\end{document}
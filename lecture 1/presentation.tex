\documentclass[a4paper, 11pt]{beamer}

\usepackage{polski}
\usepackage[utf8]{inputenc}

\mode<presentation> {
	\usetheme{Frankfurt}
	\setbeamercovered{transparent}
	\usecolortheme{default}
}

\title{Ekonometria Finansowa}
\subtitle{Zarys zagadnień związanych z rynkami finansowymi}
\author{mgr Paweł Jamer\thanks{pawel.jamer@gmail.com}}

\begin{document}

	\begin{frame}
		\titlepage
	\end{frame}
	
	\section{Rynki finansowe}

	\begin{frame}{Rynek finansowy}
		\begin{block}{\textbf{Rynek finansowy}}
			Miejsce dokonywania kupna lub sprzedaży instrumentów finansowych.
		\end{block}
		\begin{block}{\textbf{Instrument finansowy} (ustawa o rachunkowości)}
			Kontrakt zawarty pomiędzy dwiema stronami, powodujący powstanie aktywów
			finansowych u jednej ze stron i zobowiązania finansowego lub instrumentu
			kapitałowego u drugiej ze stron.
		\end{block}
		\textbf{Rynki finansowe:}
		\begin{itemize}
			\item pieniężny,
			\item kapitałowy,
			\item walutowy,
			\item instrumentów pochodnych.
		\end{itemize}
	\end{frame}

	\begin{frame}{Funkcje rynków finansowych}
		\begin{block}{\textbf{Funkcja podstawowa}}
			Pośrednictwo w transferze funduszy od uczestników posiadających ich nadwyżkę,
			do tych, którzy zgłaszają na nie popyt.
		\end{block}
		\textbf{Funkcje:}
		\begin{itemize}
			\item mobilność i mobilizacja kapitału,
			\item alokacja kapitału,
			\item transformacja kapitału,
			\item wycena wartości firmy i kapitału.
		\end{itemize}
	\end{frame}
	
	\begin{frame}{Klasyfikacja rynków finansowych}
		\begin{block}{\textbf{Kryterium zapadalności}}
			\begin{itemize}
				\item \textbf{Rynek pieniężny:}
				\begin{itemize}
					\item termin realizacji praw nie dłuższy niż rok,
					\item wyłącznie wierzytelności pieniężne.
				\end{itemize}
				\item \textbf{Rynek kapitałowy:}
				\begin{itemize}
					\item termin realizacji praw powyżej roku,
					\item instrumenty o charakterze wierzycielskim lub majątkowym.
				\end{itemize}
			\end{itemize}
		\end{block}
		\begin{block}{\textbf{Termin transakcji}}
			\begin{itemize}
				\item \textbf{Rynek kasowy} --- dostawa i zapłata dokonywane są w tym samym czasie (transfer kapitału).
				\item \textbf{Rynek terminowy} --- między zapłatą i dostawą występuje różnica w czasie (transfer ryzyka).
			\end{itemize}
		\end{block}
		\textbf{Inne kryteria:} skala transakcji, przedmiot transakcji itd.
	\end{frame}
	
	\begin{frame}{Struktura rynku kapitałowego}
		\begin{block}{\textbf{Rynek kapitałowy}}
			\begin{itemize}
				\item \textbf{Rynek prywatny:}
				\begin{itemize}
					\item mniej niż 150 oznaczonych inwestorów,
					\item brak nadzoru KNF.
				\end{itemize}
				\item \textbf{Rynek publiczny (papierów wartościowych):}
				\begin{itemize}
					\item co najmniej 150 inwestorów lub inwestorzy nieoznaczeni,
					\item nadzór KNF.
				\end{itemize}
			\end{itemize}
		\end{block}
		\begin{block}{\textbf{Rynek publiczny}}
			\begin{itemize}
				\item \textbf{Rynek pierwotny:}
				\begin{itemize}
					\item emitent-inwestor,
					\item oferta publiczna,
					\item obowiązki informacyjne.
				\end{itemize}
				\item \textbf{Rynek wtórny:}
				\begin{itemize}
					\item inwestor-inwestor,
					\item obowiązki informacyjne.
				\end{itemize}
			\end{itemize}
		\end{block}
	\end{frame}
	
	\begin{frame}{Struktura rynku wtórnego}
		\begin{block}{\textbf{Rynek wtórny}}
			\begin{itemize}
				\item \textbf{Rynek regulowany:}
				\begin{itemize}
					\item działa w sposób stały,
					\item podlega nadzorowi KNF,
					\item zapewnia powszechny i równy dostęp do informacji rynkowej,
					\item zapewnia jednakowe warunki nabywania i zbywania instrumentów finansowych.
				\end{itemize}
				\item \textbf{Rynek alternatywny:}
				\begin{itemize}
					\item zapewnia koncentrację podaży i popytu na instrumenty finansowe,
					\item nie gwarantuje powszechnego i równego dostępu do informacji rynkowej,
					\item nie gwarantuje jednakowych warunków nabywania i zbywania instrumentów finansowych.
				\end{itemize}
			\end{itemize}
		\end{block}
	\end{frame}
	
	\begin{frame}{Uczestnicy rynków finansowych}
		\begin{columns}[onlytextwidth]
			\begin{column}{0.5\textwidth}
				\textbf{Poszukujący kapitału:}
				\begin{itemize}
					\item kredytobiorcy,
					\item emitenci.
				\end{itemize}
			\end{column}
			\begin{column}{0.5\textwidth}
				\textbf{Dysponujący kapitałem:}
				\begin{itemize}
					\item depozytariusze,
					\item inwestorzy,
					\item banki,
					\item zakłady ubezpieczeń,
					\item fundusze,
					\item venture capital,
					\item private equity.
				\end{itemize}
			\end{column}
		\end{columns}
	\end{frame}
	
	\section{Instrumenty finansowe}
	
	\begin{frame}{Instrumenty rynku pieniężnego}
		\begin{block}{\textbf{Bon skarbowy}}
			Instrument finansowy emitowany przez skarb państwa w celu pozyskania
			kapitału na krótki okres. Instrument zerokuponowy, wolny od ryzyka.
		\end{block}
		\begin{block}{\textbf{Bon komercyjny}}
			Instrument finansowy emitowany przez przedsiębiorstwo w celu pozyskania
			kapitału na krótki okres. Z reguły zerokuponowy, charakteryzuje się
			wyższym ryzykiem niż bon skarbowy.
		\end{block}
		\begin{block}{\textbf{Certyfikat depozytowy}}
			Instrument finansowy emitowany przez bank. Depozyt, który można sprzedać
			na rynku wtórnym. Zazwyczaj instrument o podstawie odsetkowej.
		\end{block}
	\end{frame}
	
	\begin{frame}{Instrumenty rynku pieniężnego}
		\begin{block}{\textbf{Transakcja repo}}
			Strona krótka sprzedaje stronie długiej pewien instrument finansowy,
			zobowiązując się jednocześnie do odkupienia go po określonej cenie w
			określonym momencie w przyszłości.
		\end{block}
		\begin{block}{\textbf{Transakcja reverse repo}}
			Strona długa kupuje od strony krótkiej pewien instrument finansowy,
			zobowiązując się jednocześnie do odsprzedania go po określonej cenie
			w określonym momencie w przyszłości.
		\end{block}
	\end{frame}
	
	\begin{frame}{Obligacje}
		\begin{block}{\textbf{Obligacja}}
			Strona krótka jest dłużnikiem strony długiej. Zobowiązuje się do
			wykupu obligacji, polegającego na zapłaceniu wartości nominalnej
			obligacji oraz odsetek, jeżeli występują.
		\end{block}
		\textbf{Parametry:}
		\begin{itemize}
			\item wartość nominalna,
			\item termin wykupu,
			\item oprocentowanie,
			\item termin płacenia odsetek.
		\end{itemize}
	\end{frame}
	
	\begin{frame}{Akcje}
		\begin{block}{\textbf{Akcja}}
			Instrument udziałowy emitowany przez spółkę akcyjną.
			\begin{itemize}
				\item Spółka emitując akcje uzyskuje nowy kapitał.
				\item Inwestor (akcjonariusz) kupując akcje staje się współwłaścicielem spółki.
			\end{itemize}
		\end{block}
		\textbf{Prawa akcjonariusza:}
		\begin{itemize}
			\item prawo własności,
			\item prawo głosu na walnym zgromadzeniu akcjonariuszy,
			\item prawo do dywidendy,
			\item prawo poboru,
			\item prawo do udziału w masie likwidacyjnej,
			\item prawo do kontroli zarządzania spółką.
		\end{itemize}
	\end{frame}
	
	\begin{frame}{Opcje}
		\begin{block}{\textbf{Opcja kupna (call)}}
			Prawo kupna określonej ilości instrumentu podstawowego po ustalonej cenie w ustalonym okresie.
		\end{block}
		\begin{block}{\textbf{Opcja sprzedaży (put)}}
			Prawo sprzedaży określonej ilości instrumentu podstawowego po ustalonej cenie w ustalonym okresie.
		\end{block}
		\textbf{Parametry:}
		\begin{itemize}
			\item ilość instrumentu podstawowego,
			\item cena wykonania,
			\item okres w którym może dojść do transakcji.
		\end{itemize}
	\end{frame}
	
	\begin{frame}{Kontrakty terminowe}
		\begin{block}{\textbf{Kontrakt terminowy}}
			Zobowiązanie dwóch stron do przeprowadzenia transakcji kupna-sprzedaży ustalonej ilości
			instrumentu podstawowego w ustalonym dniu po ustalonej cenie.
		\end{block}
		\textbf{Kontrakty:}
		\begin{itemize}
			\item \textbf{forward} --- występują poza giełdą,
			\item \textbf{futures} --- występują na giełdzie.
		\end{itemize}
	\end{frame}
	
	\begin{frame}{Kontrakty swap}
		\begin{block}{\textbf{Kontrakt swap}}
			Instrument finansowy, w którym każda ze stron dokonuje w ustalonych terminach w przyszłości
			serii płatności na rzecz drugiej strony kontraktu, przy czym przynajmniej jedna seria płatności
			zależy od wartości indeksu podstawowego.
		\end{block}
		\textbf{Parametry:}
		\begin{itemize}
			\item indeks podstawowy,
			\item terminy rozliczenia.
		\end{itemize}
	\end{frame}
	
	\begin{frame}{Indeksy giełdowe}
		\begin{block}{\textbf{Indeks giełdowy}}
			Wielkość zmieniająca się w trakcie obrotów na rynku, określana na podstawie wielkości
			charakteryzujących instrumenty finansowe występujące na rynku.
		\end{block}
		\textbf{Funkcje:}
		\begin{itemize}
			\item informuje o sytuacji na giełdzie,
			\item stanowi punkt odniesienia przy ocenie efektywności różnych metod inwestowania,
			\item jest instrumentem bazowym dla instrumentów pochodnych,
			\item jest przybliżeniem portfela rynkowego.
		\end{itemize}
	\end{frame}
	
	\section{Stopy zwrotu}
	
	\begin{frame}{Stopy zwrotu}
		\begin{block}{\textbf{Stopa zwrotu}}
			Niech $p_t$ oznacza kurs pewnego instrumentu finansowego w pewnej chwili
			czasu $t,$ wówczas stopą zwrotu tego instrumentu w chwili $t$ nazwiemy
			wielkość \[
				r_t = \frac{p_t - p _{t-1}}{p_{t-1}}.
			\]
		\end{block}
		\begin{block}{\textbf{Logarytmiczna stopa zwrotu}}
			Niech $p_t$ oznacza kurs pewnego instrumentu finansowego w pewnej chwili
			czasu $t,$ wówczas logarytmiczną stopą zwrotu tego instrumentu w chwili
			$t$ nazwiemy wielkość \[
				r_t = \log\left(\frac{p_t}{p_{t-1}}\right).
			\]
		\end{block}
	\end{frame}
	
	\begin{frame}{Charakterystyka rozkładów stóp zwrotu}
		\textbf{Rozkłady stóp zwrotu często charakteryzuje:}
		\begin{itemize}
			\item \textbf{leptokurtyczność} --- wartości rozkładu są bardziej skoncentrowane
				wokół średniej niż przy rozkładzie normalnym;
			\item \textbf{występowanie grubych ogonów} --- wartości ekstremalne są bardziej prawdopodobne
				niż w rozkładzie normalnym;
			\item \textbf{skośność} --- pozytywne (negatywne) wartości ekstremalne są bardziej
				prawdopodobne niż negatywne (pozytywne) wartości ekstremalne;
			\item \textbf{grupowanie zmienności} --- duża zmienność w chwili $t$ zwiększa
				prawdopodobieństwo wystąpienia dużej zmienności w chwili $t+1.$
		\end{itemize}
	\end{frame}
	
	\begin{frame}{Modelowanie stóp zwrotu}
		\begin{alert}{\textbf{Uwaga. }}
			Wiele stosowanych w praktyce na rynkach finansowych modeli zakłada normalność rozkładu stopy zwrotu.
			Założenie to dla odpowiednio dużej próby może okazać się prawdziwe na mocy Centralnego Twierdzenia
			Granicznego. W ogólnym przypadku należy jednak podchodzić do niego z dużym dystansem.
		\end{alert}
		\\~\\
		\textbf{Alternatywne rozkłady służące modelowaniu stóp zwrotu:}
		\begin{itemize}
			\item rozkład t-Studenta,
			\item rozkład skośny t-Studenta,
			\item uogólniony rozkład normalny / błędu (GGD / GED).
		\end{itemize}
	\end{frame}

	\section*{}

	\begin{frame}
		\center
		\Huge \bfseries
		Pytania?
	\end{frame}

	\begin{frame}
		\center
		\Huge \bfseries
		Dziękuję za uwagę!
	\end{frame}

\end{document}
\documentclass[a4paper, 11pt]{beamer}

\usepackage{polski}
\usepackage[utf8]{inputenc}
\usepackage{units}

\mode<presentation> {
	\usetheme{Frankfurt}
	\setbeamercovered{transparent}
	\usecolortheme{default}
}

\title{Ekonometria Finansowa}
\subtitle{Dekompozycja szeregów czasowych}
\author{mgr Paweł Jamer\thanks{pawel.jamer@gmail.com}}
\institute[KEiS SGGW]{
	Doktorant, Katedra Ekonometrii i Statystyki SGGW\newline
	Ekspert ds. Modelowania Danych, Polskie Technologie\newline
	Konsultant Zewnętrzny, Polkomtel
}

\begin{document}

	\begin{frame}
		\titlepage
	\end{frame}
	
	\section{Definicje}
	
	\begin{frame}{Klasyczny model dekompozycji szeregu czasowego}
		Niestacjonarny szereg czasowy $X_t$ zapisać możemy w postaci \[
			X_t = m_t + s_t + Y_t,
		\]
		gdzie
		\begin{itemize}
			\item $m_t$ --- składowa trendu,
			\item $s_t$ --- składowa sezonowa spełniająca warunek \[
					\left(\exists d>0\right) s_{t+d} = s_d,
				\]
			\item $Y_t$ --- proces SSS taki, że $\mathbb{E}\left(Y_t\right)=0.$
		\end{itemize}
	\end{frame}
	
	\begin{frame}{Ekonomiczny model dekompozycji szeregu czasowego}
		Ekonomiści dokonują czasami dekompozycji niestacjonarnego szeregu czasowego
		$X_t$ w sposób następujący \[
			X_t = m_t + s_t + c_t + Y_t,
		\]
		gdzie
		\begin{itemize}
			\item $m_t$ --- składowa trendu,
			\item $s_t$ --- składowa sezonowa,
			\item $c_t$ --- składowa wahań cyklicznych (koniunkturalnych),
			\item $Y_t$ --- proces SSS taki, że $\mathbb{E}\left(Y_t\right)=0.$
		\end{itemize}
		\begin{alert}{\textbf{Uwaga.}}
			Składowa wahań cyklicznych jest rozpatrywana albo łącznie z trendem albo
			łącznie ze składową przypadkową.
		\end{alert}
	\end{frame}

	\section{Estymacja składowej trendu}
	
	\begin{frame}{Przegląd metod}
		W celu wyodrębnienia składowej trendu można:
		\begin{itemize}
			\item wykorzystać model trendu,
			\item wykorzystać średnie kroczące,
			\item dokonać różnicowania szeregu czasowego,
			\item ...
		\end{itemize}
	\end{frame}
	
	\begin{frame}{Model trendu}
		\framesubtitle{Definicja}
		\begin{block}{\textbf{Model trendu}}
			Modelem trendu szeregu czasowego $X_t$ nazwiemy model postaci \[
				X_t = f\left(t\right) + \epsilon_t,
			\]
			gdzie
			\begin{itemize}
				\item $f\left(t\right)$ --- funkcja trendu,
				\item $\epsilon_t$ --- składnik losowy.
			\end{itemize}
		\end{block}
		
		\begin{alert}{\textbf{Uwaga.}}
			Nie czyni się żadnych założeń co do postaci składnika losowego w modelu
			trendu.
		\end{alert}
	\end{frame}
	
	\begin{frame}{Model trendu}
		\framesubtitle{Dobór funkcji trendu}
		Funkcję trendu wybiera się bazując na:
		\begin{itemize}
			\item przesłankach teoretycznych,
			\item analizie danych empirycznych,
			\item badaniu przyrostów szeregów,
			\item porównaniu kilku możliwych funkcji.
		\end{itemize}
		\begin{alert}{\textbf{Uwaga.}}
			W praktyce jako funkcję trendu przyjmuje się najczęściej funkcję 
			wielomianową. W szczególności funkcję liniową, będącą szczególnym
			przypadkiem funkcji wielomianowej.
		\end{alert}
	\end{frame}
	
	\begin{frame}{Średnia ruchoma}
		\framesubtitle{Definicja}
		\begin{block}{\textbf{Średnia ruchoma}}
			Średnią ruchomą rzędu $q$ szeregu czasowego $X_t$ nazwiemy statystykę 
				postaci \[
				\overline{X}_t =
					\frac{1}{2q + 1} \sum_{\left|\tau\right| \leq q} X_{t-\tau},
				\mbox{ gdy }q\mbox{ nieparzyste}
			\]
			lub \[
				\overline{X}_t =
					\frac{1}{2q} \left(
						\frac{1}{2} X_{t-q} +
						\sum_{\left|\tau\right| < q} X_{t-\tau} +
						\frac{1}{2} X_{t+q}
					\right),
					\mbox{ gdy }q\mbox{ parzyste}.
			\]
		\end{block}
		\begin{alert}{\textbf{Uwaga.}}
			Przy odpowiednio dobranej wartości parametru $q$ zachodzi \[
				m_t \approx \overline{X}_t.
			\]
		\end{alert}
	\end{frame}
	
	\begin{frame}{Średnia ruchoma}
		\framesubtitle{Dobór rzędu średniej ruchomej}
		Przyjmijmy $s_t = 0.$ Zauważmy, że \begin{eqnarray*}
			\overline{X}_t & = &
				\frac{1}{2q + 1}
				\sum_{\left|\tau\right| \leq q} \left(m_{t-\tau} + Y_{t-\tau}\right)\\
			\overline{X}_t & = &
				\frac{1}{2q + 1} \sum_{\left|\tau\right| \leq q} m_{t-\tau} +
				\frac{1}{2q + 1} \sum_{\left|\tau\right| \leq q} Y_{t-\tau}
		\end{eqnarray*}
		Wówczas:
		\begin{itemize}
			\item dla małych $q:$
				$\frac{1}{2q + 1} \sum_{\left|\tau\right| \leq q} m_{t-\tau} 
				\approx m_t,$
			\item dla dużych $q:$
				$\frac{1}{2q + 1} \sum_{\left|\tau\right| \leq q} Y_{t-\tau} \approx 0.$
		\end{itemize}
		\begin{alert}{\textbf{Wniosek.}}
			Rząd średniej ruchomej powinien być na tyle duży, by zmarginalizować 
			znaczenie składowej $Y_t$ oraz na tyle mały, by nie zaburzyć zależności
			$\overline{X}_t \approx m_t.$
		\end{alert}
	\end{frame}
	
	\begin{frame}{Różnicowanie}
		Przyjmijmy $s_t = 0.$ Niech trend ma postać \[
			m_t = \sum_{j=0}^{k} \beta_j t^j.
		\]
		Wówczas operacja $k$-krotnego różnicowania \[
			\Delta^k X_t
		\] pozwala wyeliminować trend z szeregu czasowego $X_t.$
		\\~\\
		\begin{alert}{\textbf{Wniosek.}}
			W celu wyeliminowania trendu o postaci wielomianowej można zastosować
			operację różnicowania szeregu.
		\end{alert}
	\end{frame}

	\section{Estymacja składowej sezonowej}
	
	\begin{frame}{Przegląd metod}
		W celu wyodrębnienia składowej sezonowej można:
		\begin{itemize}
			\item wykorzystać metodę wskaźników,
			\item zastosować modele z sezonowymi zmiennymi binarnymi,
			\item dokonać analizy harmonicznej szeregu czasowego,
			\item ...
		\end{itemize}

		\begin{alert}{\textbf{Uwaga.}}
			Zastsowanie metody wskaźników oraz modeli ze zmiennymi binarnymi wymaga
			uprzedniego ustalenia okresu $s$ szeregu czasowego. Zastsowanie analiy
			harmonicznej nie wiąże się z koniecznością uprzedniego określenia okresu
			$s.$
		\end{alert}
	\end{frame}
	
	\begin{frame}{Model wskaźników}
		\framesubtitle{Wskaźnik sezonowości}
		\begin{block}{\textbf{Wskaźnik sezonowości}}
			Wskaźnikiem sezonowości szeregu czasowego $X_t$ nazwiemy statystykę 
			postaci\[
				S_t = \frac{
					\sum_{\tau=0}^{\frac{T}{s}-1} X_{t + s \tau}
				}{
					\sum_{\tau=0}^{\frac{T}{s}-1} \hat{X}_{t + s \tau}
				},
			\] gdzie
			\begin{itemize}
				\item $\hat{X}_t$ --- estymator $X_t.$
			\end{itemize}
		\end{block}
		\begin{alert}{\textbf{Uwaga.}}
			Wartość $\hat{X}_t$ uzyskać można poprzez zastosowanie modelu trendu lub
			wyznaczenie średniej ruchomej.
		\end{alert}
	\end{frame}
	
	\begin{frame}{Model wskaźników}
		\framesubtitle{Normalizacja}
		Od wskaźników sezonowości oczekuje się, aby spełniały zależność \[
			\sum_{j=1}^{s} S_j = s.
		\]
		W związku z czym dokonuje się ich normalizacji \[
			S_j^n = \frac{s}{\sum_{j=1}^{s} S_j} \cdot S_j.
		\]
	\end{frame}
	
	\begin{frame}{Model wskaźników}
		\framesubtitle{Składowa sezonowa}
		Składową sezonową w oparciu o znormalizowane wskaźniki sezonowości
		oblicza się ostatecznie jako \[
			s_t = S_{t \bmod s} \overline{X} - \overline{X},
		\] gdzie
		\begin{itemize}
			\item $\overline{X} = \sum_{t=1}^{T} X_t.$
		\end{itemize}
	\end{frame}
	
	\begin{frame}{Model z sezonowymi zmiennymi binarnymi}
		\framesubtitle{Definicja}
		\begin{block}{\textbf{Model z sezonowymi zmiennymi binarnymi}}
			Modelem z sezonowymi zmiennymi binarnymi szeregu czasowego $X_t$ o 
			sezonowości $s$ nazwiemy model postaci \[
				X_t = f\left(t\right) + \sum_{i=1}^{s} \delta_i z_{i,t} + \epsilon_t,
			\]
			gdzie
			\begin{itemize}
				\item $f\left(t\right)$ --- funkcja trendu,
				\item $z_{i,t}$ --- zmienna przyjmująca wartość $1$ dla $i$-tego momentu
					okresu oraz $0$ w pozostałych przypadkach,
				\item $\delta_i$ --- stały efekt sezonowy dla $i$-tego momentu okresu,
				\item $\epsilon_t$ --- składnik losowy.
			\end{itemize}
		\end{block}
	\end{frame}
	
	\begin{frame}{Model z sezonowymi zmiennymi binarnymi}
		\framesubtitle{Estymacja}
		\begin{alert}{\textbf{Uwaga.}}
			Modelu z sezonowymi zmiennymi binarnymi nie da się estymować z 
			wykorzystaniem klasycznej metody najmniejszych kwadratów z uwagi na
			występowanie nim binarnych zmiennych sezonowych.
		\end{alert}
		\\~\\
		W celu estymacji modelu z sezonowymi zmiennymi binarnymi za pomocą KMNK
		usuwa się z niego zmienną $z_{s,t}.$ Stały efekt sezonowy dla $s$-tego
		momentu okresu wylicza się wówczas z wzoru \[
			\hat{\delta}_s = -\sum_{j=1}^{s-1} \hat{\delta}_j.
		\]
	\end{frame}
	
	\section{Algorytmy dekompozycji}

	\begin{frame}{Algorytm Holta-Wintersa}
		\framesubtitle{Algorytm}
		Algorytm Holta-Wintersa dla klasycznego modelu dekompozycji szeregu 
		czasowego o znanym okresie $d$ to układ rekurencyjnych równań 
		postaci: \begin{eqnarray*}
			\hat{m}_{t+1} & = &
				\alpha \left(X_{t+1} - \hat{s}_{t+1-d}\right) +
				\left(1 - \alpha\right) \left( \hat{m}_t + \hat{m}^{\prime}_t\right),\\
			\hat{m}^{\prime}_{t+1} & = &
				\beta \left(\hat{m}_{t+1} - \hat{m}_t\right) +
				\left(1 - \beta\right) \hat{m}^{\prime}_t,\\
			\hat{s}_{t+1} & = &
				\gamma \left(X_{t+1} - \hat{m}_{t+1}\right) +
				\left(1 - \gamma\right) \hat{s}_{t+1-d}.
		\end{eqnarray*}
		\begin{alert}{\textbf{Intuicja.}}
			Algorytm Holta-Wintersa to iteracyjna metoda jednoczesnej estymacji 
			składowej trendu oraz składowej sezonowej w klasycznym modelu 
			dekompozycji szeregu czasowego.
		\end{alert}
	\end{frame}
	
	\begin{frame}{Algorytm Holta-Wintersa}
		\framesubtitle{Wartości początkowe}
		Wartości inicjalne algorytmu Holta-Wintersa obliczane są na podstawie 
		pierwszych $d+1$ obserwacji analizowanego szeregu czasowego $X_t$ i 
		przyjmują postać: \begin{eqnarray*}
			\hat{m}_{d+1} & = & X_{d+1},\\
			\hat{m}^{\prime}_{d+1} & = & \frac{X_{d+1} - X_1}{d},\\
			\hat{s}_{\tau} & = & X_{\tau} -
				\left(X_1 - \hat{m}^{\prime}_{d+1}\left(\tau - 1\right)\right), 
				\tau = 1, 2, \ldots, d+1.
		\end{eqnarray*}
	\end{frame}
	
	\begin{frame}{Algorytm Holta-Wintersa}
		\framesubtitle{Uwagi}
		\begin{alert}{\textbf{Uwaga.}}
			Algorytm Holta-Wintersa wykorzystuje w swoim działaniu zarówno wiedzę 
			wynikającą z postaci klasycznego modelu dekompozycji, jak też wiedzę 
			zawartą w analizowanej trajektorii szeregu czasowego $X_t.$
		\end{alert}
		\\~\\
		\begin{alert}{\textbf{Uwaga.}}
			Doboru wartości parametrów $\alpha,$ $\beta$ oraz $\gamma$ dokonuje się
			poprzez minimalizację sumy \[
				\sum_{t=d+2}^{T} \left(X_t - \hat{X}_t\right)^2,
			\] której często dokonuje się w praktyce poprzez poszukiwanie po siatce.
		\end{alert}
		\\~\\
		\begin{alert}{\textbf{Uwaga.}}
			Przyjęcie $\alpha \in \left(0, 1\right)$ oraz $\beta = 0$ i $\gamma = 0$
			sprowadza algorytm Holta-Wintersa do algorytmu estymacji składowej trendu
			metodą wygładzania wykładniczego.
		\end{alert}
	\end{frame}

	\begin{frame}{Algorytm pięciokrokowy}
		\framesubtitle{Algorytm}
		Rozważmy klasyczny model dekompozycji szeregu czasowego $X_t$ o znanym 
		okresie $d$ oraz składowej sezonowej $s_t$ takie, że \[
			\sum_{t=1}^{d} s_t = 0.
		\]
		Algorytm pięciokrokowy dekompozycji szeregu czsowego $X_t$ polega na 
		wykonaniu poniższych pięciu kroków
		\begin{enumerate}
			\item Wstępnej estymacji trendu $m_t.$
			\item Estymacji składowej sezonowej $s_t.$
			\item Desezonalizacji szeregu $X_t$.
			\item Parametrycznej estymacji składowej trendu $m_t.$
			\item Dopasowaniu modelu procesu stacjonarnego $Y_t.$
		\end{enumerate}
	\end{frame}

	\section*{}

	\begin{frame}
		\center
		\Huge \bfseries
		Pytania?
	\end{frame}

	\begin{frame}
		\center
		\Huge \bfseries
		Dziękuję za uwagę!
	\end{frame}

\end{document}